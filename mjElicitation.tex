\RequirePackage[l2tabu, orthodox]{nag}
\documentclass[version=3.21, pagesize, twoside=off, bibliography=totoc, DIV=calc, fontsize=12pt, a4paper]{scrartcl}
%Permits to copy eg x ⪰ y ⇔ v(x) ≥ v(y) from PDF to unicode data, and to search. From pdfTeX users manual. See https://tex.stackexchange.com/posts/comments/1203887.
	\input glyphtounicode
	\pdfgentounicode=1
%Latin Modern has more glyphs than Computer Modern, such as diacritical characters. fntguide commands to load the font before fontenc, to prevent default loading of cmr.
	\usepackage{lmodern}
%Encode resulting accented characters correctly in resulting PDF, permits copy from PDF.
	\usepackage[T1]{fontenc}
%UTF8 seems to be the default in recent TeX installations, but not all, see https://tex.stackexchange.com/a/370280.
	\usepackage[utf8]{inputenc}
%Provides \newunicodechar for easy definition of supplementary UTF8 characters such as → or ≤ for use in source code.
	\usepackage{newunicodechar}
%Text Companion fonts, much used together with CM-like fonts. Provides \texteuro and commands for text mode characters such as \textminus, \textrightarrow, \textlbrackdbl.
	\usepackage{textcomp}
%St Mary’s Road symbol font, used for ⟦ = \llbracket. The \SetSymbolFont command avoids spurious warnings, but also some valid ones, see https://tex.stackexchange.com/a/106719/.
	%\usepackage{stmaryrd}\SetSymbolFont{stmry}{bold}{U}{stmry}{m}{n}
%Solves bug in lmodern, https://tex.stackexchange.com/a/261188; probably useful only for unusually big font sizes; and probably better to use exscale instead. Note that the authors of exscale write against this trick.
	%\DeclareFontShape{OMX}{cmex}{m}{n}{
		%<-7.5> cmex7
		%<7.5-8.5> cmex8
		%<8.5-9.5> cmex9
		%<9.5-> cmex10
	%}{}
	%\SetSymbolFont{largesymbols}{normal}{OMX}{cmex}{m}{n}
%More symbols (such as \sum) available in bold version, see https://github.com/latex3/latex2e/issues/71. In article mode (but not in presentation mode), also hides some potentially useful warnings such as when using $\bm{\llbracket}$, see stmaryrd in this document (not sure why).
	\DeclareFontShape{OMX}{cmex}{bx}{n}{%
	   <->sfixed*cmexb10%
	   }{}
	\SetSymbolFont{largesymbols}{bold}{OMX}{cmex}{bx}{n}
%https://english.stackexchange.com/questions/93008
	\usepackage[super]{nth}
%For small caps also in italics, see https://tex.stackexchange.com/questions/32942/italic-shape-needed-in-small-caps-fonts, https://tex.stackexchange.com/questions/284338/italic-small-caps-not-working.
	\usepackage{slantsc}
	\AtBeginDocument{%
		%“Since nearly no font family will contain real italic small caps variants, the best approach is to substitute them by slanted variants.” -- slantsc doc
		%\DeclareFontShape{T1}{lmr}{m}{scit}{<->ssub*lmr/m/scsl}{}%
		%There’s no bold small caps in Latin Modern, we switch to Computer Modern for bold small caps, see https://tex.stackexchange.com/a/22241
		%\DeclareFontShape{T1}{lmr}{bx}{sc}{<->ssub*cmr/bx/sc}{}%
		%\DeclareFontShape{T1}{lmr}{bx}{scit}{<->ssub*cmr/bx/scsl}{}%
	}
%Warn about missing characters.
	\tracinglostchars=2
%Nicer tables: provides \toprule, \midrule, \bottomrule.
	%\usepackage{booktabs}
%For new column type X which stretches; can be used together with booktabs, see https://tex.stackexchange.com/a/97137. “tabularx modifies the widths of the columns, whereas tabular* modifies the widths of the inter-column spaces.” Loads array.
	%\usepackage{tabularx}
%math-mode version of "l" column type. Requires \usepackage{array}.
	%\usepackage{array}
	%\newcolumntype{L}{>{$}l<{$}}
%Provides \xpretocmd and loads etoolbox which provides \apptocmd, \patchcmd, \newtoggle… Also loads xparse, which provides \NewDocumentCommand and similar commands intended as replacement of \newcommand in LaTeX3 for defining commands (see https://tex.stackexchange.com/q/98152 and https://github.com/latex3/latex2e/issues/89).
	\usepackage{xpatch}
%ntheorem doc says: “empheq provides an enhanced vertical placement of the endmarks”; must be loaded before ntheorem. Loads the mathtools package, which loads and fixes some bugs in amsmath and provides \DeclarePairedDelimiter. amsmath is considered a basic, mandatory package nowadays (Grätzer, More Math Into LaTeX).
	\usepackage[ntheorem]{empheq}
%Package frenchb asks to load natbib before babel-french. Package hyperref asks to load natbib before hyperref.
	\usepackage{natbib}

\newtoggle{LCpres}
	\newtoggle{LCart}
	\newtoggle{LCposter}
	\makeatletter
	\@ifclassloaded{beamer}{
		\toggletrue{LCpres}
		\togglefalse{LCart}
		\togglefalse{LCposter}
		\wlog{Presentation mode}
	}{
		\@ifclassloaded{tikzposter}{
			\toggletrue{LCposter}
			\togglefalse{LCpres}
			\togglefalse{LCart}
			\wlog{Poster mode}
		}{
			\toggletrue{LCart}
			\togglefalse{LCpres}
			\togglefalse{LCposter}
			\wlog{Article mode}
		}
	}
	\makeatother%

%Language options ([french, english]) should be on the document level (last is main); except with tikzposter: put [french, english] options next to \usepackage{babel} to avoid warning. beamer uses the \translate command for the appendix: omitting babel results in a warning, see https://github.com/josephwright/beamer/issues/449. Babel also seems required for \refname.
	\iftoggle{LCpres}{
		\usepackage{babel}
	}{
	}
	%\frenchbsetup{AutoSpacePunctuation=false}
%listings (1.7) does not allow multi-byte encodings. listingsutf8 works around this only for characters that can be represented in a known one-byte encoding and only for \lstinputlisting. Other workarounds: use literate mechanism; or escape to LaTeX (but breaks alignment).
	%\usepackage{listings}
	%\lstset{tabsize=2, basicstyle=\ttfamily, escapechar=§, literate={é}{{\'e}}1}
%I favor acro over acronym because the former is more recently updated (2018 VS 2015 at time of writing); has a longer user manual (about 40 pages VS 6 pages if not counting the example and implementation parts); has a command for capitalization; and acronym suffers a nasty bug when ac used in section, see https://tex.stackexchange.com/q/103483 (though this might be the fault of the silence package and might be solved in more recent versions, I do not know) and from a bug when used with cleveref, see https://tex.stackexchange.com/q/71364. However, loading it makes compilation time (one pass on this template) go from 0.6 to 1.4 seconds, see https://bitbucket.org/cgnieder/acro/issues/115.
	\usepackage{acro}
	\DeclareAcronym{MJ}{short=MJ, long={Majority Judgment}}



\iftoggle{LCpres}{
	%I favor fmtcount over nth because it is loaded by datetime anyway; and fmtcount warns about possible conflicts when loaded after nth.
	\usepackage{fmtcount}
	%For nice input of date of presentation. Must be loaded after the babel package. Has possible problems with srcletter: https://golatex.de/verwendung-von-babel-und-datetime-in-scrlttr2-schlaegt-fehlt-t14779.html.
	\usepackage[nodayofweek]{datetime}
}{
}
%For presentations, Beamer implicitely uses the pdfusetitle option. ntheorem doc says to load hyperref “before the first use of \newtheorem”. autonum doc mandates option hypertexnames=false. I want to highlight links only if necessary for the reader to recognize it as a link, to reduce distraction. In presentations, this is already taken care of by beamer (https://tex.stackexchange.com/a/262014). If using colorlinks=true in a presentation, see https://tex.stackexchange.com/q/203056. Crashes the first compilation with tikzposter, just compile again and the problem disappears, see https://tex.stackexchange.com/q/254257.
\makeatletter
\iftoggle{LCpres}{
	\usepackage{hyperref}
}{
	\usepackage[hypertexnames=false, pdfusetitle, linkbordercolor={1 1 1}, citebordercolor={1 1 1}, urlbordercolor={1 1 1}]{hyperref}
	%https://tex.stackexchange.com/a/466235
	\pdfstringdefDisableCommands{%
		\let\thanks\@gobble
	}
}
\makeatother
%urlbordercolor is used both for \url and \doi, which I think shouldn’t be colored, and for \href, thus might want to color manually when required. Requires xcolor.
	\NewDocumentCommand{\hrefblue}{mm}{\textcolor{blue}{\href{#1}{#2}}}
%hyperref doc says: “Package bookmark replaces hyperref’s bookmark organization by a new algorithm (...) Therefore I recommend using this package”.
	\usepackage{bookmark}
%Need to invoke hyperref explicitly to link to line numbers: \hyperlink{lintarget:mylinelabel}{\ref*{lin:mylinelabel}}, with \ref* to disable automatic link. Also see https://tex.stackexchange.com/q/428656 for referencing lines from another document.
	%\usepackage{lineno}
	%\NewDocumentCommand{\llabel}{m}{\hypertarget{lintarget:#1}{}\linelabel{lin:#1}}
	%\setlength\linenumbersep{9mm}
%For complex authors blocks. Seems like authblk wants to be later than hyperref, but sooner than silence. See https://tex.stackexchange.com/q/475513 for the patch to hyperref pdfauthor.
	\ExplSyntaxOn
	\seq_new:N \g_oc_hrauthor_seq
	\NewDocumentCommand{\addhrauthor}{m}{
		\seq_gput_right:Nn \g_oc_hrauthor_seq { #1 }
	}
	%Should be \NewExpandableDocumentCommand, but this is not yet provided by my version of xparse
	\DeclareExpandableDocumentCommand{\hrauthor}{}{
		\seq_use:Nn \g_oc_hrauthor_seq {,~}
	}
	\ExplSyntaxOff
	{
		\catcode`#=11\relax
		\gdef\fixauthor{\xpretocmd{\author}{\addhrauthor{#2}}{}{}}%
	}
	\iftoggle{LCart}{
		\usepackage{authblk}
		\renewcommand\Affilfont{\small}
		\fixauthor
		\AtBeginDocument{
		    \hypersetup{pdfauthor={\hrauthor}}
		}
	}{
	}
%I do not use floatrow, because it requires an ugly hack for proper functioning with KOMA script (see scrhack doc). Instead, the following command centers all floats (using \centering, as the center environment adds space, http://texblog.net/latex-archive/layout/center-centering/), and I manually place my table captions above and figure captions below their contents (https://tex.stackexchange.com/a/3253).
	\makeatletter
	\g@addto@macro\@floatboxreset\centering
	\makeatother
%Permits to customize enumeration display and references
	%\nottoggle{LCpres}{
		%\usepackage{enumitem} %follow list environments by a string to customize enumeration, example: \begin{description}[itemindent=8em, labelwidth=!] or \begin{enumerate}[label=({\roman*}), ref={\roman*}].
	%}{
	%}
%Provides \Centering, \RaggedLeft, and \RaggedRight and environments Center, FlushLeft, and FlushRight, which allow hyphenation. With tikzposter, seems to cause 1=1 to be printed in the middle of the poster.
	%\usepackage{ragged2e}
%To typeset units by closely following the “official” rules.
	%\usepackage[strict]{siunitx}
%Turns the doi provided by some bibliography styles into URLs. However, uses old-style dx.doi url (see 3.8 DOI system Proxy Server technical details, “Users may resolve DOI names that are structured to use the DOI system Proxy Server (https://doi.org (current, preferred) or earlier syntax http://dx.doi.org).”, https://www.doi.org/doi_handbook/3_Resolution.html). The patch solves this.
	\usepackage{doi}
	\makeatletter
	\patchcmd{\@doi}{http://dx.doi.org}{https://doi.org}{}{}
	\makeatother
%Makes sure upper case greek letters are italic as well.
	\usepackage{fixmath}
%Provides \mathbb; obsoletes latexsym (see http://tug.ctan.org/macros/latex/base/latexsym.dtx). Relatedly, \usepackage{eucal} to change the mathcal font and \usepackage[mathscr]{eucal} (apparently equivalent to \usepackage[mathscr]{euscript}) to supplement \mathcal with \mathscr. This last option is not very useful as both fonts are similar, and the intent of the authors of eucal was to provide a replacement to mathcal (see doc euscript). Also provides \mathfrak for supplementary letters.
	\usepackage{amsfonts}
%Provides a beautiful (IMHO) \mathscr and really different than \mathcal, for supplementary uppercase letters. But there is no bold version. Alternative: mathrsfs (more slanted), but when used with tikzposter, it warns about size substitution, see https://tex.stackexchange.com/q/495167.
	\usepackage[scr]{rsfso}
%Multiple means to produce bold math: \mathbf, \boldmath (defined to be \mathversion{bold}, see fntguide), \pmb, \boldsymbol (all legacy, from LaTeX base and AMS), \bm (the most recommended one), \mathbold from package fixmath (I don’t see its advantage over \boldsymbol).
%“The \boldsymbol command is obtained preferably by using the bm package, which provides a newer, more powerful version than the one provided by the amsmath package. Generally speaking, it is ill-advised to apply \boldsymbol to more than one symbol at a time.” — AMS Short math guide. “If no bold font appears to be available for a particular symbol, \bm will use ‘poor man’s bold’” — bm. It is “best to load the package after any packages that define new symbol fonts” – bm. bm defines \boldsymbol as synonym to \bm. \boldmath accesses the correct font if it exists; it is used by \bm when appropriate. See https://tex.stackexchange.com/a/10643 and https://github.com/latex3/latex2e/issues/71 for some difficulties with \bm.
	\usepackage{bm}
	\nottoggle{LCpres}{
	%https://ctan.org/pkg/amsmath recommends ntheorem, which supersedes amsthm, which corrects the spacing of proclamations and allows for theoremstyle. Option standard loads amssymb and latexsym. Must be loaded after amsmath (from ntheorem doc). From cleveref doc, “ntheorem is fully supported and even recommended”; says to load cleveref after ntheorem. When used with tikzposter, warns about size substitution for the lasy (latexsym) font when using \url, because ntheorem loads latexsym; relatedly (but not directly related to ntheorem), size substitution warning with the cmex font happens when loading amsmath and using \url. According to https://tex.stackexchange.com/q/535950, ntheorem “seems essentially unmaintaned and has severe problems”, but I use it anyway because it is very handy. Yields “! LaTeX Error: Something's wrong--perhaps a missing \item.” if some theorem follows thebibliography.
		\usepackage[thmmarks, amsmath, standard, hyperref]{ntheorem}
		%empheq doc says to do this after loading ntheorem
		\usetagform{default}
	%Provides \cref. Unfortunately, cref fails when the language is French and referring to a label whose name contains a colon (https://tex.stackexchange.com/q/83798). Use \cref{sec\string:intro} to work around this. cleveref should go “laster” than hyperref.
		\usepackage{cleveref}
	}{
	}
	\nottoggle{LCposter}{
	%Equations get numbers iff they are referenced. Loading order should be “amsmath → hyperref → cleveref → autonum”, according to autonum doc. Use this in preference to the showonlyrefs option from mathtools, see https://tex.stackexchange.com/q/459918 and autonum doc. See https://tex.stackexchange.com/a/285953 for the etex line. Incompatible with my version of tikzposter (produces “! Improper \prevdepth”).
		\expandafter\def\csname ver@etex.sty\endcsname{3000/12/31}\let\globcount\newcount
		\usepackage{autonum}
	}{
	}
%Also loaded by tikz.
	%\usepackage{xcolor}
\iftoggle{LCpres}{
	\usepackage{tikz}
	%\usetikzlibrary{babel, matrix, fit, plotmarks, calc, trees, shapes.geometric, positioning, plothandlers, arrows, shapes.multipart}
}{
}
%Vizualization, on top of TikZ
	%\usepackage{pgfplots}
	%\pgfplotsset{compat=1.14}
\usepackage{graphicx}
	\graphicspath{{graphics/}}

%Provides \printlength{length}, useful for debugging.
	%\usepackage{printlen}
	%\uselengthunit{mm}

\iftoggle{LCpres}{
	\usepackage{appendixnumberbeamer}
	%I have yet to see anyone actually use these navigation symbols; let’s disable them
	\setbeamertemplate{navigation symbols}{} 
	\usepackage{preamble/beamerthemeParisFrance}
	\setcounter{tocdepth}{10}
}{
}

%Do not use the displaymath environment: use equation. Do not use the eqnarray or eqnarray* environments: use align(*). This improves spacing. (See l2tabu or amsldoc.)


%Requires package xcolor.
%\definecolor{ao(english)}{rgb}{0.0, 0.5, 0.0}
\NewDocumentCommand{\commentOC}{m}{\textcolor{blue}{\small$\big[$OC: #1$\big]$}}
%Requires package babel and option [french]. According to babel doc, need two braces around \selectlanguage to make the changes really local.
\NewDocumentCommand{\commentOCf}{m}{\textcolor{blue}{{\small\selectlanguage{french}$\big[$OC : #1$\big]$}}}
\NewDocumentCommand{\commentYM}{m}{\textcolor{red}{\small$\big[$YM: #1$\big]$}}
\NewDocumentCommand{\commentYMf}{m}{\textcolor{red}{{\small\selectlanguage{french}$\big[$YM : #1$\big]$}}}

\bibliographystyle{abbrvnat}
\NewDocumentCommand{\possessivecite}{mO{}}{\citeauthor{#1}’s \citeyearpar[#2]{#1}}

%https://tex.stackexchange.com/a/467188, https://tex.stackexchange.com/a/36088 - uncomment if one of those symbols is used.
%\DeclareFontFamily{U} {MnSymbolD}{}
%\DeclareFontShape{U}{MnSymbolD}{m}{n}{
%  <-6> MnSymbolD5
%  <6-7> MnSymbolD6
%  <7-8> MnSymbolD7
%  <8-9> MnSymbolD8
%  <9-10> MnSymbolD9
%  <10-12> MnSymbolD10
%  <12-> MnSymbolD12}{}
%\DeclareFontShape{U}{MnSymbolD}{b}{n}{
%  <-6> MnSymbolD-Bold5
%  <6-7> MnSymbolD-Bold6
%  <7-8> MnSymbolD-Bold7
%  <8-9> MnSymbolD-Bold8
%  <9-10> MnSymbolD-Bold9
%  <10-12> MnSymbolD-Bold10
%  <12-> MnSymbolD-Bold12}{}
%\DeclareSymbolFont{MnSyD} {U} {MnSymbolD}{m}{n}
%\DeclareMathSymbol{\ntriplesim}{\mathrel}{MnSyD}{126}
%\DeclareMathSymbol{\nlessgtr}{\mathrel}{MnSyD}{192}
%\DeclareMathSymbol{\ngtrless}{\mathrel}{MnSyD}{193}
%\DeclareMathSymbol{\nlesseqgtr}{\mathrel}{MnSyD}{194}
%\DeclareMathSymbol{\ngtreqless}{\mathrel}{MnSyD}{195}
%\DeclareMathSymbol{\nlesseqgtrslant}{\mathrel}{MnSyD}{198}
%\DeclareMathSymbol{\ngtreqlessslant}{\mathrel}{MnSyD}{199}
%\DeclareMathSymbol{\npreccurlyeq}{\mathrel}{MnSyD}{228}
%\DeclareMathSymbol{\nsucccurlyeq}{\mathrel}{MnSyD}{229}
%\DeclareFontFamily{U} {MnSymbolA}{}
%\DeclareFontShape{U}{MnSymbolA}{m}{n}{
%  <-6> MnSymbolA5
%  <6-7> MnSymbolA6
%  <7-8> MnSymbolA7
%  <8-9> MnSymbolA8
%  <9-10> MnSymbolA9
%  <10-12> MnSymbolA10
%  <12-> MnSymbolA12}{}
%\DeclareFontShape{U}{MnSymbolA}{b}{n}{
%  <-6> MnSymbolA-Bold5
%  <6-7> MnSymbolA-Bold6
%  <7-8> MnSymbolA-Bold7
%  <8-9> MnSymbolA-Bold8
%  <9-10> MnSymbolA-Bold9
%  <10-12> MnSymbolA-Bold10
%  <12-> MnSymbolA-Bold12}{}
%\DeclareSymbolFont{MnSyA} {U} {MnSymbolA}{m}{n}
%%Rightwards wave arrow: ↝. Alternative: \rightsquigarrow from amssymb, but it’s uglier
%\DeclareMathSymbol{\rightlsquigarrow}{\mathrel}{MnSyA}{160}

%03B3 Greek Small Letter Gamma
\newunicodechar{γ}{\gamma}
%03B4 Greek Small Letter Delta
\newunicodechar{δ}{\delta}
%2115 Double-Struck Capital N
\newunicodechar{ℕ}{\mathbb{N}}
%211D Double-Struck Capital R
\newunicodechar{ℝ}{\mathbb{R}}
%21CF Rightwards Double Arrow with Stroke
\newunicodechar{⇏}{\nRightarrow}
%21D2 Rightwards Double Arrow
\newunicodechar{⇒}{\ensuremath{\Rightarrow}}
%21D4 Left Right Double Arrow
\newunicodechar{⇔}{\Leftrightarrow}
%21DD Rightwards Squiggle Arrow
\newunicodechar{⇝}{\rightsquigarrow}
%2205 Empty Set
\newunicodechar{∅}{\emptyset}
%2212 Minus Sign
\newunicodechar{−}{\ifmmode{-}\else\textminus\fi}
%2227 Logical And
\newunicodechar{∧}{\land}
%2228 Logical Or
\newunicodechar{∨}{\lor}
%2229 Intersection
\newunicodechar{∩}{\cap}
%222A Union
\newunicodechar{∪}{\cup}
%2260 Not Equal To (handy also as text in informal writing)
\newunicodechar{≠}{\ensuremath{\neq}}
%2264 Less-Than or Equal To
\newunicodechar{≤}{\leq}
%2265 Greater-Than or Equal To
\newunicodechar{≥}{\geq}
%2270 Neither Less-Than nor Equal To
\newunicodechar{≰}{\nleq}
%2271 Neither Greater-Than nor Equal To
\newunicodechar{≱}{\ngeq}
%2272 Less-Than or Equivalent To
\newunicodechar{≲}{\lesssim}
%2273 Greater-Than or Equivalent To
\newunicodechar{≳}{\gtrsim}
%2274 Neither Less-Than nor Equivalent To – also, from MnSymbol: \nprecsim, a more exact match to the Unicode symbol; and \npreccurlyeq, too small
\newunicodechar{≴}{\not\preccurlyeq}
%2275 Neither Greater-Than nor Equivalent To
\newunicodechar{≵}{\not\succcurlyeq}
%2279 Neither Greater-Than nor Less-Than – requires MnSymbol; also \nlessgtr from txfonts/pxfonts, \ngtreqless from MnSymbol (but much higher), \ngtrless from MnSymbol (a more exact match to the Unicode symbol); for incomparability (not matching this Unicode symbol), may also consider \ntriplesim from MnSymbol,\nparallelslant from fourier, \between from mathabx, or ⋈
\newunicodechar{≹}{\ngtreqlessslant}
%227A Precedes
\newunicodechar{≺}{\prec}
%227B Succeeds
\newunicodechar{≻}{\succ}
%227C Precedes or Equal To
\newunicodechar{≼}{\preccurlyeq}
%227D Succeeds or Equal To
\newunicodechar{≽}{\succcurlyeq}
%227E Precedes or Equivalent To
\newunicodechar{≾}{\precsim}
%227F Succeeds or Equivalent To
\newunicodechar{≿}{\succsim}
%2280 Does Not Precede
\newunicodechar{⊀}{\nprec}
%2281 Does Not Succeed
\newunicodechar{⊁}{\nsucc}
%2286
\newunicodechar{⊆}{\subseteq}
%22B2 Normal Subgroup Of – using \vartriangleleft from amsfonts, which goes well with \trianglelefteq, \ntriangleright, and so on, also from amsfonts; another possibility is \lhd from latexsym, which seems visually equivalent to \vartriangleleft from amsfonts; latexsym also has ⊴=\unlhd, but doesn’t have a symbol for ⊴. Other related symbols: \triangleleft from latesym package is too small; fdsymbol provides \triangleleft=\medtriangleleft and \vartriangleleft=\smalltriangleleft; MnSymbol provides \medtriangleleft and \vartriangleleft=\lessclosed=\lhd which are smaller than \vartriangleleft from amsfont; \vartriangleleft from mathabx (p. 67), looks different (wider); also \vartriangleleft from boisik (p. 69) looks still different; \vartriangleleft=\lhd from stix are smaller. Oddly enough, \triangleright appears as the LMMathItalic12-Regular font whereas \rhd appears as LASY10 and \vartriangleright appears as MSAM10.
\newunicodechar{⊲}{\vartriangleleft}
%22B3 Contains as Normal Subgroup (also: 25B7 White right-pointing triangle or 25B9 White right-pointing small triangle)
\newunicodechar{⊳}{\vartriangleright}
%22B4 Normal Subgroup of or Equal To
\newunicodechar{⊴}{\trianglelefteq}
%22B5 Contains as Normal Subgroup or Equal To
\newunicodechar{⊵}{\trianglerighteq}
%22C8 Bowtie
\newunicodechar{⋈}{\bowtie}
%22EA Not Normal Subgroup Of
\newunicodechar{⋪}{\ntriangleleft}
%22EB Does Not Contain As Normal Subgroup
\newunicodechar{⋫}{\ntriangleright}
%22EC Not Normal Subgroup of or Equal To
\newunicodechar{⋬}{\ntrianglelefteq}
%22ED Does Not Contain as Normal Subgroup or Equal
\newunicodechar{⋭}{\ntrianglerighteq}
%25A1 White Square
\newunicodechar{□}{\Box}
%27E6 Mathematical Left White Square Bracket – requires stmaryrd (alternative: \text{\textlbrackdbl}, but ugly if used in an italicized text such as a theorem)
\newunicodechar{⟦}{\llbracket}
%27E7 Mathematical Right White Square Bracket
\newunicodechar{⟧}{\rrbracket}
%27FC Long Rightwards Arrow from Bar
\newunicodechar{⟼}{\longmapsto}
%2AB0 Succeeds Above Single-Line Equals Sign
\newunicodechar{⪰}{\succeq}
%301A Left White Square Bracket
\newunicodechar{〚}{\textlbrackdbl}
%301B Right White Square Bracket
\newunicodechar{〛}{\textrbrackdbl}
%→ is defined by default as \textrightarrow, which is invalid in math mode. Same thing for the three other commands. Using \DeclareUnicodeCharacter instead of \newunicodechar because the latter warns about the previous definition.
%← Leftwards Arrow
\DeclareUnicodeCharacter{2190}{\ifmmode\leftarrow\else\textleftarrow\fi}
%→ Rightwards Arrow
\DeclareUnicodeCharacter{2192}{\ifmmode\rightarrow\else\textrightarrow\fi}
%¬ Not Sign
\DeclareUnicodeCharacter{00AC}{\ifmmode\lnot\else\textlnot\fi}
%… Horizontal Ellipsis
\DeclareUnicodeCharacter{2026}{\ifmmode\dots\else\textellipsis\fi}
%× Multiplication Sign
\DeclareUnicodeCharacter{00D7}{\ifmmode\times\else\texttimes\fi}
%Permits to really obtain a straight quote when typing a straight quote; potentially dangerous, see https://tex.stackexchange.com/a/521999
\catcode`\'=\active
\DeclareUnicodeCharacter{0027}{\ifmmode^\prime\else\textquotesingle\fi}


\NewDocumentCommand{\R}{}{ℝ}
\NewDocumentCommand{\N}{}{ℕ}
%\mathscr is rounder than \mathcal.
\NewDocumentCommand{\powerset}{m}{\mathscr{P}(#1)}
%Powerset without zero.
\NewDocumentCommand{\powersetz}{m}{\mathscr{P}^*(#1)}
%https://tex.stackexchange.com/a/45732, works within both \set and \set*, same spacing than \mid (https://tex.stackexchange.com/a/52905).
\NewDocumentCommand{\suchthat}{}{\;\ifnum\currentgrouptype=16 \middle\fi|\;}
%Integer interval.
\NewDocumentCommand{\intvl}{m}{⟦#1⟧}
%Allows for \abs and \abs*, which resizes the delimiters.
\DeclarePairedDelimiter\abs{\lvert}{\rvert}
\DeclarePairedDelimiter\card{\lvert}{\rvert}
\DeclarePairedDelimiter\floor{\lfloor}{\rfloor}
\DeclarePairedDelimiter\ceil{\lceil}{\rceil}
%Perhaps should use U+2016 ‖ DOUBLE VERTICAL LINE here?
\DeclarePairedDelimiter\norm{\lVert}{\rVert}
%From mathtools. Better than using the package braket because braket introduces possibly undesirable space. Then: \begin{equation}\set*{x \in \R^2 \suchthat \norm{x}<5}\end{equation}.
\DeclarePairedDelimiter\set{\{}{\}}
\DeclareMathOperator*{\argmax}{arg\,max}
\DeclareMathOperator*{\argmin}{arg\,min}

%UTR #25: Unicode support for mathematics recommend to use the straight form of phi (by default, given by \phi) rather than the curly one (by default, given by \varphi), and thus use \phi for the mathematical symbol and not \varphi. I however prefer the curly form because the straight form is too easy to mix up with the symbol for empty set.
\let\phi\varphi

%The amssymb solution.
%\NewDocumentCommand{\restr}{mm}{{#1}_{\restriction #2}}
%Another acceptable solution.
%\NewDocumentCommand{\restr}{mm}{{#1|}_{#2}}
%https://tex.stackexchange.com/a/278631; drawback being that sometimes the text collides with the line below.
\NewDocumentCommand\restr{mm}{#1\raisebox{-.5ex}{$|$}_{#2}}


%Decision Theory (MCDA and SC)
\NewDocumentCommand{\allalts}{}{\mathscr{A}}
\NewDocumentCommand{\allcrits}{}{\mathscr{C}}
\NewDocumentCommand{\alts}{}{A}
\NewDocumentCommand{\dm}{}{i}
\NewDocumentCommand{\allF}{}{\mathscr{F}}
\NewDocumentCommand{\allvoters}{}{\mathscr{N}}
\NewDocumentCommand{\voters}{}{N}
\NewDocumentCommand{\prof}{}{\boldsymbol{P}}
\NewDocumentCommand{\linors}{}{\mathscr{L}(\allalts)}
%Thanks to https://tex.stackexchange.com/q/154549
	%\makeatletter
	%\def\@myRgood@#1#2{\mathrel{R^X_{#2}}}
	%\def\myRgood{\@ifnextchar_{\@myRgood@}{\mathrel{R^X}}}
	%\makeatother
\NewDocumentCommand{\pref}{}{\succ}
\NewDocumentCommand{\prefi}{O{i}}{\succ_{#1}}

%Deliberated Judgment
\NewDocumentCommand{\allargs}{}{S^*}
\NewDocumentCommand{\args}{}{S}
\NewDocumentCommand{\ar}{}{s}
\NewDocumentCommand{\allprops}{}{T}
\NewDocumentCommand{\prop}{}{t}
\NewDocumentCommand{\ileadsto}{}{⇝}
\NewDocumentCommand{\ibeatse}{}{⊳_\exists}
\NewDocumentCommand{\nibeatse}{}{⋫_\exists}
\NewDocumentCommand{\ibeatsst}{}{⊳_\forall}
\NewDocumentCommand{\nibeatsst}{}{⋫_\forall}
\NewDocumentCommand{\mleadsto}{O{\eta}}{⇝_{#1}}
\NewDocumentCommand{\mbeats}{O{\eta}}{⊳_{#1}}
\NewDocumentCommand{\ibeatseinv}{}{⊳_\exists^{-1}}

%Logic
\NewDocumentCommand{\ltru}{}{\texttt{T}}
\NewDocumentCommand{\lfal}{}{\texttt{F}}

\NewDocumentCommand{\dbar}{}{\overline{\delta}}
\NewDocumentCommand{\reddbar}{}{{\color{red}\overline{\delta}}}
\NewDocumentCommand{\second}{}{\prime\prime}
\NewDocumentCommand{\Pbar}{}{\overline{P}}
\NewDocumentCommand{\fmaj}{o}{
	\IfNoValueTF {#1}{f_\mathit{maj}}{f^{#1}_\mathit{maj}}
}
\NewDocumentCommand{\Fmaj}{o}{
	\IfNoValueTF {#1}{F_\mathit{maj}}{F^{#1}_\mathit{maj}}
}
\NewDocumentCommand{\fmajbar}{o}{
	\IfNoValueTF {#1}{\overline{f}_\mathit{maj}}{\overline{f}^{#1}_\mathit{maj}}
}
\NewDocumentCommand{\Fmajbar}{o}{
	\IfNoValueTF {#1}{\overline{F}_\mathit{maj}}{\overline{F}^{#1}_\mathit{maj}}
}
\NewDocumentCommand{\med}{}{\text{med}}

%\NewDocumentCommand{\tikzmark}{m}{%
	\tikz[overlay, remember picture, baseline=(#1.base)] \node (#1) {};%
}

\newlength{\GraphsDNodeSep}
\setlength{\GraphsDNodeSep}{7mm}
\tikzset{/GraphsD/dot/.style={
	shape=circle, fill=black, inner sep=0, minimum size=1mm
}}

% MCDA Drawing Sorting
\newlength{\MCDSCatHeight}
\setlength{\MCDSCatHeight}{6mm}
\newlength{\MCDSAltHeight}
\setlength{\MCDSAltHeight}{4mm}
%separation between two vertical alts
\newlength{\MCDSAltSep}
\setlength{\MCDSAltSep}{2mm}
\newlength{\MCDSCatWidth}
\setlength{\MCDSCatWidth}{3cm}
\newlength{\MCDSAltWidth}
\setlength{\MCDSAltWidth}{2.5cm}
\newlength{\MCDSEvalRowHeight}
\setlength{\MCDSEvalRowHeight}{6mm}
\newlength{\MCDSAltsToCatsSep}
\setlength{\MCDSAltsToCatsSep}{1.5cm}
\newcounter{MCDSNbAlts}
\newcounter{MCDSNbCats}
\newlength{\MCDSArrowDownOffset}
\setlength{\MCDSArrowDownOffset}{0mm}
\tikzset{/MCD/S/alt/.style={
	shape=rectangle, draw=black, inner sep=0, minimum height=\MCDSAltHeight, minimum width=\MCDSAltWidth
}}
\tikzset{/MCD/S/pref/.style={
	shape=ellipse, draw=gray, thick
}}
\tikzset{/MCD/S/cat/.style={
	shape=rectangle, draw=black, inner sep=0, minimum height=\MCDSCatHeight, minimum width=\MCDSCatWidth
}}
\tikzset{/MCD/S/evals matrix/.style={
	matrix, row sep=-\pgflinewidth, column sep=-\pgflinewidth, nodes={shape=rectangle, draw=black, inner sep=0mm, text depth=0.5ex, text height=1em, minimum height=\MCDSEvalRowHeight, minimum width=12mm}, nodes in empty cells, matrix of nodes, inner sep=0mm, outer sep=0mm, row 1/.style={nodes={draw=none, minimum height=0em, text height=, inner ysep=1mm}}
}}

%Git
\newlength{\GitDCommitSep}
\setlength{\GitDCommitSep}{13mm}
\tikzset{/GitD/commit/.style={
	shape=rectangle, draw, minimum width=4em, minimum height=0.6cm
}}
\tikzset{/GitD/branch/.style={
	shape=ellipse, draw, red
}}
\tikzset{/GitD/head/.style={
	shape=ellipse, draw, fill=yellow
}}

%Social Choice
\tikzset{/SCD/profile matrix/.style={
	matrix of math nodes, column sep=3mm, row sep=2mm, nodes={inner sep=0.5mm, anchor=base}
}}
\tikzset{/SCD/rank-profile matrix/.style={
	matrix of math nodes, column sep=3mm, row sep=2mm, nodes={anchor=base}, column 1/.style={nodes={inner sep=0.5mm}}, row 1/.style={nodes={inner sep=0.5mm}}
}}
\tikzset{/SCD/rank-vector/.style={
	draw, rectangle, inner sep=0, outer sep=1mm
}}
\tikzset{/SCD/isolated rank-vector/.style={
	draw, matrix of math nodes, column sep=3mm, inner sep=0, matrix anchor=base, nodes={anchor=base, inner sep=.33em}, ampersand replacement=\&
}}

% GUI
\tikzset{/GUID/button/.style={
	rectangle, very thick, rounded corners, draw=black, fill=black!40%, top color=black!70, bottom color=white
}}

% Logger objects
\tikzset{/loggerD/main/.style={
	shape=rectangle, draw=black, inner sep=1ex, minimum height=7mm
}}
\tikzset{/loggerD/helper/.style={
	shape=rectangle, draw=black, dashed, minimum height=7mm
}}
\tikzset{/loggerD/helper line/.style={
	<->, draw, dotted
}}

% Beliefs
\tikzset{/BeliefsD/attacker/.style={
	shape=rectangle, draw, minimum size=8mm
}}
\tikzset{/BeliefsD/supporter/.style={
	shape=circle, draw
}}


%\usepackage{xstring}
\usepackage{pgffor}

\ExplSyntaxOn
\NewDocumentCommand{\extractlast}{O{.}mo}
 {
  \seq_set_split:Nnn \l_stroobants_string_seq { #1 } { #2 }
  \IfNoValueTF { #3 }
   {
    \seq_item:Nn \l_stroobants_string_seq { -1 }
   }
   {
    \tl_set:Nx #3 { \seq_item:Nn \l_stroobants_string_seq { -1 } }
   }
 }
\seq_new:N \l_stroobants_string_seq
\ExplSyntaxOff

\newbool{jdocRefTypewriter}
\newbool{jdocRefShowPackage}
\newbool{jdocRefShowField}
\newbool{jdocRefShowMethod}
\newbool{jdocRefShowParameters}
\pgfkeys{
	/jdocRef/.is family,
	/jdocRef/.cd,
	full text/.estore in = \jdocRefFullText,
	parameters text/.estore in = \jdocRefParametersText,
	base url/.estore in = \jdocRefBaseUrl,
	parameters style/.estore in = \jdocRefParametersStyle,
	typewriter/.is if = jdocRefTypewriter,
	show package/.is if = jdocRefShowPackage,
	show field/.is if = jdocRefShowField,
	show method/.is if = jdocRefShowMethod,
	show parameters/.is if = jdocRefShowParameters,
	default/.style = {
		full text =,
		parameters text=,
		typewriter = true,
		show package = false,
		show field = true,
		show method = true,
		show parameters = false,
	},
}

%Thx https://tex.stackexchange.com/a/34318.
%Approach: define the following jdocRef macros. AtSign (starts with at); Module (before unique / or empty); Method (after unique # if has ( or empty); Field (after unique # if no ( or empty if no #); FQName (between / and # or after / or before # or everything); FQNameSlashes (FQName with slashes instead of dots); Class (between last dot and #).
%The main argument specifies everything needed to build the URL and the default appearance (@ or no @…), except possibly for default global variables such as base url. The default appearance can be changed with keys.
%Notation: I considered using double colon to separate class from method (https://stackoverflow.com/a/59418704), and writing parameters as, for example, Random::doubles(origin: double, bound: double), following UML notation. But to be coherent, must then use the double colon to separate package elements and class from inner classes, for example, java::util::Random and HTML::Attribute::HEIGHT, which is not readable. Seems more reasonable to follow closely the Javadoc notation (after all, we refer to Java classes, not language-neutral classes). The only addition being when I need the parameter names, which I write as Random#doubles(double origin, double bound). Note that it is necessary to distinguish static method calls: Math.sqrt(16) and object methods: String#toUpperCase(). Using a dot in the latter case is unsatisfactory, as it lets the reader wonder whether this is a static method that is called. Using a dash in the first case is unsatisfactory, as, when the parameter is bound, we want to illustrate a specific call, not refer to a method in general.
%TODO permit to show a dot or a dash (default to dash). Permit to omit entirely the parenthesis, to allow for displaying Math.sqrt(16) or Math.sqrt(Math.PI). Or better, allow to specify the content (but this may be a jdocref again).
\newcommand{\jdocRef}[2][]{%
	\pgfkeys{/jdocRef/.cd, default, #1}%
	\edef\jdocRefFull{#2}%
	\IfBeginWith{#2}{@}{%
		\edef\jdocRefAtSign{@}%
		\StrBehind{#2}{@}[\jdocRefWithoutAt]%
	}{%
		\edef\jdocRefAtSign{}%
		\edef\jdocRefWithoutAt{#2}%
	}%
	\IfSubStr{#2}{/}{%
	}{%
		\typeout{Missing slash!}
	}%
	\StrBefore{\jdocRefWithoutAt}{/}[\jdocRefBeforeFirstSlash]%
	\StrBehind{\jdocRefWithoutAt}{/}[\jdocRefAfterFirstSlash]%
	\IfSubStr{\jdocRefAfterFirstSlash}{/}{%
		\edef\jdocRefModule{\jdocRefBeforeFirstSlash}%
		\StrBefore{\jdocRefAfterFirstSlash}{/}[\jdocRefPackage]%
		\StrBehind{\jdocRefAfterFirstSlash}{/}[\jdocRefClassAndRest]%
	}{%
		\edef\jdocRefModule{}%
		\edef\jdocRefPackage{\jdocRefBeforeFirstSlash}%
		\edef\jdocRefClassAndRest{\jdocRefAfterFirstSlash}%
	}%
	\StrSubstitute{\jdocRefPackage}{.}{/}[\jdocRefPackageSlashes]%
	\IfSubStr{#2}{\#}{%
		\StrBehind{#2}{\#}[\jdocRefAfterSharp]%
		\IfSubStr{\jdocRefAfterSharp}{(}{%
			\StrBehind{#2}{\#}[\jdocRefMethodWithParameters]%
		}{%
			\edef\jdocRefMethodWithParameters{}%
		}%
	}{%
		\edef\jdocRefMethodWithParameters{}%
	}%
	\StrSubstitute{\jdocRefMethodWithParameters}{ }{}[\jdocRefMethodWithParametersWithoutSpaces]%
	\StrBefore{\jdocRefMethodWithParameters}{(}[\jdocRefMethodWithoutParameters]%
	\IfSubStr{#2}{\#}{%
		\StrBehind{#2}{\#}[\jdocRefAfterSharp]%
		\IfSubStr{\jdocRefAfterSharp}{(}{%
			\edef\jdocRefField{}%
		}{%
			\StrBehind{#2}{\#}[\jdocRefField]%
		}%
	}{%
		\edef\jdocRefField{}%
	}%
	\IfSubStr{\jdocRefClassAndRest}{\#}{%
		\StrBefore{\jdocRefClassAndRest}{\#}[\jdocRefClass]%
	}{%
		\edef\jdocRefClass{\jdocRefClassAndRest}%
	}%
	%
	%\jdocRefPrintAll%
	%
	\edef\jdocRefTarget{\jdocRefBaseUrl}%
	\IfEq{\jdocRefModule}{}{%
	}{%
		\appto\jdocRefTarget{\jdocRefModule}%
		\appto\jdocRefTarget{/}%
	}%
	\appto\jdocRefTarget{\jdocRefPackageSlashes}%
	\appto\jdocRefTarget{/}%
	\appto\jdocRefTarget{\jdocRefClass}%
	\appto\jdocRefTarget{.html}%
	\IfEq{\jdocRefField}{}{%
	}{%
		\appto\jdocRefTarget{\#\jdocRefField}%
	}%
	\IfEq{\jdocRefMethodWithParametersWithoutSpaces}{}{%
	}{%
		\IfEq{\jdocRefParametersStyle}{hyphen}{%
			\StrSubstitute{\jdocRefMethodWithParametersWithoutSpaces}{(}{-}[\jdocRefMethodWithParametersWithoutSpacesWithHyphens]%
			\StrSubstitute{\jdocRefMethodWithParametersWithoutSpacesWithHyphens}{,}{-}[\jdocRefMethodWithParametersWithoutSpacesWithHyphens]%
			\StrSubstitute{\jdocRefMethodWithParametersWithoutSpacesWithHyphens}{)}{-}[\jdocRefMethodWithParametersWithoutSpacesWithHyphens]%
			\appto\jdocRefTarget{\#\jdocRefMethodWithParametersWithoutSpacesWithHyphens}%
		}{%
			\appto\jdocRefTarget{\#\jdocRefMethodWithParametersWithoutSpaces}%
		}%
	}%
	%
	\edef\jdocRefLinkText{}%
	\appto\jdocRefLinkText{\jdocRefAtSign}%
	\ifbool{jdocRefShowPackage}{%
		\appto\jdocRefLinkText{\jdocRefPackage}%
		\appto\jdocRefLinkText{.}%
	}{%
	}%
	\appto\jdocRefLinkText{\jdocRefClass}%
	\ifbool{jdocRefShowField}{%
		\IfEq{\jdocRefField}{}{%
		}{%
			\appto\jdocRefLinkText{\#\jdocRefField}%
		}%
	}{%
	}%
	\ifbool{jdocRefShowMethod}{%
		\ifbool{jdocRefShowParameters}{%
			\IfEq{\jdocRefMethodWithParameters}{}{%
			}{%
				\appto\jdocRefLinkText{\#\jdocRefMethodWithoutParameters}%
				\appto\jdocRefLinkText{(}%
				\appto\jdocRefLinkText{\jdocRefParametersText}%
				\appto\jdocRefLinkText{)}%
			}%
		}{%
			\IfEq{\jdocRefMethodWithoutParameters}{}{%
			}{%
				\appto\jdocRefLinkText{\#\jdocRefMethodWithoutParameters}%
				\appto\jdocRefLinkText{(}%
				\appto\jdocRefLinkText{)}%
			}%
		}%
	}{%
	}%
	%
	\IfEq{\jdocRefFullText}{}{%
		\ifbool{jdocRefTypewriter}{%
			\href{\jdocRefTarget}{\texttt{\jdocRefLinkText}}%
		}{%
			\href{\jdocRefTarget}{\jdocRefLinkText}%
		}%
	}{%
		\href{\jdocRefTarget}{\jdocRefFullText}%
	}%
}

\newcommand{\jdocRefPrintAll}{%
	\begin{description}
		\item[BaseUrl] \foreach \sfx in {BaseUrl}{\csname jdocRef\sfx\endcsname}
		\foreach \sfx in {Full, AtSign, Module, MethodWithParameters, MethodWithParametersWithoutSpaces, MethodWithoutParameters, Field, Package, PackageSlashes, Class}{\item[\sfx] \csname jdocRef\sfx\endcsname}
	\end{description}%
}

\DeclareExpandableDocumentCommand{\jdocRefSevenBaseUrl}{}{https\string://docs.oracle.com/javase/7/docs/api/}
\DeclareExpandableDocumentCommand{\jdocRefEightBaseUrl}{}{https\string://docs.oracle.com/javase/8/docs/api/}
\DeclareExpandableDocumentCommand{\jdocRefNineBaseUrl}{}{https\string://docs.oracle.com/javase/9/docs/api/}
\DeclareExpandableDocumentCommand{\jdocRefTenBaseUrl}{}{https\string://docs.oracle.com/javase/10/docs/api/}
\DeclareExpandableDocumentCommand{\jdocRefElevenBaseUrl}{}{https\string://docs.oracle.com/en/java/javase/11/docs/api/}
\DeclareExpandableDocumentCommand{\jdocRefTwelveBaseUrl}{}{https\string://docs.oracle.com/en/java/javase/12/docs/api/}
\DeclareExpandableDocumentCommand{\jdocRefEESevenBaseUrl}{}{https\string://docs.oracle.com/javaee/7/api/}
\DeclareExpandableDocumentCommand{\jdocRefEEEightBaseUrl}{}{https\string://javaee.github.io/javaee-spec/javadocs/}

\DeclareExpandableDocumentCommand{\jdocRefBaseUrl}{}{\jdocRefElevenBaseUrl}
\DeclareExpandableDocumentCommand{\jdocRefParametersStyle}{}{parenthesis}

%TODO Let travis compile LaTeX. Make sure it fails iff compilation fails. Take https://tex.stackexchange.com/a/48613 (perhaps consider https://tex.stackexchange.com/questions/12067/whats-the-folk-lore-on-automatic-testing-of-tex-programming) for writing integration tests. In a test folder, define a test .tex file. Read https://tex.stackexchange.com/a/161982 and improve code.

%Standard HTML RenderKit : https://docs.oracle.com/javaee/7/javaserver-faces-2-2/renderkitdocs/HTML_BASIC/javax.faces.Inputjavax.faces.Text.html
%\newcommand{\jsftag}[2][]{\jjdocRef{base url = https://docs.oracle.com/javaee/7/javaserver-faces-2-2/vdldocs-facelets/, full, prefix=:, #1}}



\usepackage{stmaryrd}
\usepackage{xcolor}
\usepackage{siunitx}
\usepackage{relsize}
\pgfplotsset{compat=1.17}
%I find these settings useful in draft mode. Should be removed for final versions.
	%Which line breaks are chosen: accept worse lines, therefore reducing risk of overfull lines. Default = 200.
		\tolerance=2000
	%Accept overfull hbox up to...
		\hfuzz=2cm
	%Reduces verbosity about the bad line breaks.
		\hbadness 5000
	%Reduces verbosity about the underful vboxes.
		\vbadness=1300

\title{Preference Elicitation under Majority Judgment}
\author{}
%\title{The title \thanks{Thanks.}}
%\author{Olivier Cailloux}
%\author{Name2}
%\affil{Université Paris-Dauphine, Université PSL, CNRS, LAMSADE, 75016 PARIS, FRANCE\\
%	\href{mailto:olivier.cailloux@dauphine.fr}{olivier.cailloux@dauphine.fr}
%}
%\author{Name3}
%\affil{Affil2}
%\hypersetup{
%	pdfsubject={},
%	pdfkeywords={},
%}

\begin{document}
\maketitle

\begin{abstract}
\acl{MJ} (\acs{MJ}) is a voting system where voters assign grades to candidates using an ordinal scale. The winner is the candidate with the highest majority-grade \textemdash which is the median of the grades received. This method has attracted increasing attention of french associations and political parties which have started to use \acs{MJ} for internal decisions or local elections. In particular LaPrimaire.org is a french association that uses \acs{MJ} to choose its candidate for the french presidential election. The vote is conducted in two rounds: in the first one the voters judge five candidates randomly picked; the five candidates with the highest medians pass at the second round as finalists and the voters are asked to judge them. Is the random selection of candidates a good elicitation technique? In this paper we explore the consequences of profile incompleteness and we question the elicitation of voters preferences.
\end{abstract}

\section{Introduction}
\label{sec:intro}
\acl{MJ} (\acs{MJ}) is a voting method proposed by \citet{Balinski2007,Balinski2011} to elect one out of $m$ candidates based on the judgments of $n$ voters. The latter express their preferences by assigning to each candidate one of the following adjectives: Excellent, Very good, Good, Average, Mediocre, Inadequate, To be rejected. Those adjectives represent a common language whose semantic is assumed to be a shared knowledge among the voters carrying thus an absolute meaning. For each candidate the median of the grades she received is computed, this is called \textit{majority-grade}. The candidate with the highest majority-grade is elected. Ties are broken by considering the majority-grade of first order: one vote associated with the majority-grade of each tied candidates is removed and their medians are recomputed. The candidate with the highest new median is elected. If there is still a tie the process is repeated until a unique winner is found. 

In the last few years \acs{MJ} has being adopted by a progressively larger number of french political parties including: Le Parti Pirate, Génération(s), LaPrimaire.org, France Insoumise and La République en Marche.
%https://www.lopinion.fr/edition/politique/en-marche-teste-elections-jugement-majoritaire-mode-scrutin-tres-201884
"Mieux Voter" \citep{MV} is a french association that promotes the use of \acs{MJ} as voting method whenever a collective choice has to be selected: public administration, associations, companies. On their website it is possible to find all the citizens lists \textendash party lists that are not affiliated to any national political party \textemdash that used \acs{MJ} to rank their candidates during the local elections of 2020. In two cases, Bordeaux et Annecy, the candidate selected using \acs{MJ} was then elected as a mayor. 

In particular, LaPrimaire.org \citep{LaPrimaire} is a french political initiative whose goal is to select an independent candidate for the french presidential election using \acs{MJ} as voting rule. The association Democratech implemented the platform for the first time in 2016 in view of the 2017 presidential elections. The number of voters who participated in the election was $10676$ during the first round (with $53383$ votes) and $32685$ during the second round (with $163425$ votes). Between May and October 2021 the process will be repeated to select the candidate who will run for the 2022 presidential elections \citep{LaPrimaire2022}.

The procedure that they adopted consists of two rounds. In the first round each voter is asked to express her judgment, using \acs{MJ}, on five random candidates. At the end of this phase the five candidates with the highest medians are considered the finalists who qualify for the second round. In the second round each voter is asked to express her judgment, using \acs{MJ}, on all the five finalists. The candidate with the best median at the end of this phase is selected as representative for the presidential election.

In this paper we analyse this elicitation process of voters preferences. In particular, we investigate the consequences of randomness when asking the voters to judge candidates. We then search for more efficient techniques both in terms of communication cost \textemdash which can be quantified as number of questions per voter \textemdash and of fairness for candidates \textemdash which reflects the idea that a potential winner should not loose for lack of information.


\subsection{Related work}

To the best of our knowledge there are no works on elicitation of voter preferences under \acs{MJ}.

Several authors studied the strengths and the weakness of this method \citep{Felsenthal2008,Laslier2018} and Balinski replied to most of the critics in an article written in french published on the Revue économique \citep{Balinski2019}.

\commentBN{Add elicitation literature}

%The idea of using the median in voting is not new, the first use can be traced back to Galton's 'middlemost' \citep{Galton1907a,Galton1907b}. More recently \citet{Bassett1999} proposed the median as a substitute for Borda's mean, advocating for its statistical robustness \textemdash which measures the sensitivity to departures from the hypothesized model. An equivalent method was proposed by James W. Bucklin in the early twentieth century \citep{Hoag1926} and it was rediscovered several times in literature for example under the names of \textit{Majoritarian Compromise} \citep{Sertel1986,Sertel1999} and \textit{Fallback Bargaining} \citep{Brams2001}. Note, also, that when the number of grades is equal to two (approve, disapprove) then \acs{MJ} is reduced to Approval Voting. 


\section{Notation}
\label{sec:complete}
Consider a finite set $N$ of voters (or judges) with $\#N=n$ and a finite set $A$ of alternatives (or competitors) with $\#A=m$. 
A \textit{common language} $\triangle = \{ \delta_1, \delta_2, \dots \}$ is a set of strictly ordered grades. It may, or may not, be finite and the notation $\delta_1 \geq \delta_2$ indicates that $\delta_1$ is a better or equivalent grade than $\delta_2$. A profile $P : A\times N \rightarrow \triangle$ is a $m$ by $n$ matrix of grades $P \in \triangle^{A \times N}$. The operator $\rho: \triangle^{N} \rightarrow \triangle^n$ defines an ordering function that given a vector of grades $P_i$ returns the vector ordered by decreasing grades.

Consider a set of alternatives $S\subseteq A$,
%where $|S|=s$ and $s \in \intvl{1,m}$ \textemdash the double brackets represent an interval in the integers. 
we denote by $P^S \in \triangle^{S \times N}$ a restriction of the profile $P$ to only the alternatives in $S$, $P^S \subseteq P$. Note that when $S=A$ then $P^S=P$.
%\commentOC{Could also write $\restr{P}{S × N}$, which leads to more generality, if restricting the users is also sometimes convenient.}

We define $f: \triangle^{N} \rightarrow \triangle$ a function that assigns to any vector of grades a final grade. Given any $S\subseteq A$, a grading function $f^S: \triangle^{S \times N} \rightarrow \triangle^S$ returns a vector of final grades by applying $f$ to every alternative in $S$.
% the \emph{middlemost} aggregation function $f$, for each vector of grades $r_i= (r_1 , \dots, r_n )$ associated to the alternative $i \in \intvl{1,m}$, returns: 
%\begin{align}
%	f(r_i) &= r_{(n+1)/2} \text{ when n is odd,} \\
%	r_{n/2} \geq f(r_i) &\geq r_{(n+2)/2} \text{ otherwise.}
%\end{align}

The \emph{majority-grade}, $f_{maj}$, is the function that associates to a vector of grades $\emptyset \neq q \in \triangle^{N'}, N' \subseteq N$ its median grade value: $f_{maj}(q) = \rho(q)_{\floor{\frac{\card{q}}{2}} + 1}$. Note that in case $\card{q}$ is even two medians could be used, but, as in \citet{Balinski2011} definition, the lower grade is picked. Given any $S\subseteq A$, by applying $f_{maj}$ to all vectors of grades associated to the alternatives in $S$ we obtain the corresponding grading function $f^S_{maj}$. Formally, $f^S_\mathit{maj}(P^S)_i = f_\mathit{maj}(P^S_i)$. The $i$-\emph{th} element of the resulting vector is the median of the ordered vector of grades associated to the $i$-\emph{th} alternative. 
%Because $f^S_{maj}(P^S)_i$ depends only on $P^S_i$ we write $f^S_{maj}(P^S_i)$. \commentOC{I am not sure I follow. You already have a notation for this, namely, $f_\mathit{maj}(P^S_i)$, isn’t it?}
Moreover, when we consider the complete profile (when $S=A$) we will write $f_{maj}(P_i)$ instead of $f^A_{maj}(P^A_i)$.
\commentOC{I doubt that the $f$ notation is actually useful, it is perhaps short enough to write what it is in full (as you do just below).}
%i.e. it corresponds to the lower middlemost.

Given any $S\subseteq A$, the winner function $F^S_{maj}:\triangle^{S \times N} \rightarrow A$ %$F^S_{maj}:\triangle^{S \times N} \rightarrow 2^A \setminus \emptyset$
is a function that selects the alternative with the highest median grade as winner. We can define it as $F^S_{maj}(P^S) = \argmax_{i\in S}\rho(P^S_i)_{\floor{\frac{n}{2}}+ 1}$ assuming it is a singleton. For brevity, we will write $F_{maj}$ when considering $S=A$. Consider the case when it is not a singleton, this means that two or more alternatives are associated with the same highest median grade $h=\max_{i\in S}\rho(P^S_i)_{\floor{\frac{n}{2}} + 1}$. In this case ties are broken by removing one $h$ grade from the vectors of grades of each tied alternative, recomputing the new median grade and repeating the process until one unique winner is found or there are no more grades to remove. When $n$ is odd, this is equivalent to take the next element after the median, i.e. the one at index $(\floor{\frac{n}{2}} + 1) +1$, if there is still a tie we then look at the previous element before the median, i.e. the one at index $(\floor{\frac{n}{2}} + 1) -1$, and keep alternating until the tie is broken or there are no more elements in the vector. When $n$ is even the process is similar but we alternate starting from the element before the median, $\floor{\frac{n}{2}}$, then the one after, $\floor{\frac{n}{2}} + 2$, and so on. \commentOC{Written as it is, I am doubtful about the value of this remark.} If after applying the mechanism there are still ties we break them using an arbitrary ordering defined on all alternatives, e.g. lexicographical order.

Similarly to the winner function, given any $x \in \intvl{1,m}$, we can define a more general \emph{selection function} $F^S_x:\triangle^{S\times N} \rightarrow 2^A \setminus \emptyset$ that selects exactly the $x$ alternatives with the $x$ highest median grades.

\subsection{Incomplete Knowledge}
In order to analyse the elicitation procedure used by LaPrimaire.org, we need to adapt the notation just described to incomplete profiles. 
\commentOC{Defining the right, full notations from the start would be more gentle for the reader.} Let $\overline{P}$ be our knowledge about the profile $P$.
The voters have full knowledge of their own judgments but we ignore them; our goal is to elicit them by questioning the voters starting from zero knowledge.
We introduce an additional \emph{Undefined} grade, $\overline{\triangle}=\triangle \cup \{\textit{Undefined}\}$. Voters cannot use this grade to express their judgment over an alternative and it does not count in the computation of the median grade as we will explain formally. We refer to the grades in $\triangle$, the ones used by voters, as "defined" grades.

The starting knowledge is represented by a matrix $m\times n$ of \textit{Undefined} grades. We define with $K_j \subseteq A$ a set of $k$ alternatives that we ask the voter $j\in N$ to judge. 
After having asked every voter in $N$ to judge $k$ candidates, we obtain an \emph{incomplete profile} $\overline{P}\in \overline{\triangle}^{A \times N}$, that is a matrix $m \times n$ of grades of which $kn$ are "defined".
Note that when $k=n$ the resulting profile $\overline{P}$ corresponds to the complete profile $P$. Let $C(\overline{P}) = \{P' \in \triangle^{m \times n} \suchthat \overline{P} \subseteq P'\}$ be the set of all completions of $\overline{P}$ obtained by substituting all \emph{Undefined} grades with "defined" ones, note that $P \in C(\overline{P})$.
\commentOC{Technically that’s not correct as $P$bar may contain some “undefined”, in which case $P$ can’t be a superset of it. (Can be left as is for now.)}
 
If $K_j=K_l, \forall j,l\in N$, i.e. \commentOC{I think that ie requires a comma afterwards, and the spacing could be improved.}\commentBN{In British English, “i.e.” and “e.g.” are not followed by a comma. Virtually all American style guides recommend to follow both “i.e.” and “e.g.” with a comma(just like if “that is” and “for example” were used instead)}
\commentOC{That some people that are not British choose to write in British English has always puzzled me. That’s a bit like choosing to count in French from France instead of French from Switzerland when one has a choice. Anyway.}\commentBN{I'm not American either. It's matter of choosing a standard and be consistent. BTW I don't have a preference for any of them.}
if we ask all voters to judge the same $k$ alternatives, then we have complete knowledge restricted to this set of candidates. As defined in \Cref{sec:complete}, we call $P^{K_j}$ the restriction of the complete profile $P$ to only the alternatives in $K_j$.
\commentOC{I wonder if this notation will be useful.}

Let $g:\overline{\triangle}^N\rightarrow \bigcup_{N' \subseteq N}\triangle^{N'}$ be a function that given an incomplete vector of grades $q \in \overline{\triangle}^N$, thus $q \subseteq N × \bar{\triangle}$, returns the vector composed only of the "defined" grades $g(q) = \restr{q}{N × \triangle}$.

%The operator $\overline{\rho}$ defines a restricting ordering function that given $\overline{P_i}$, an incomplete vector of $n$ grades, returns the correspondent complete vector restricted to its $x$ "defined" grades decreasingly ordered: $\overline{\rho}(\overline{P_i})=\rho(d(\overline{P_i}))$. \commentBN{Not sure this is necessary.}

Given any $S \subseteq A$, the \emph{majority-grade} for incomplete profile $\overline{f}^S_{maj}: (\bigcup_{x \in \intvl{1,n}}\triangle^{x})^S \rightarrow \triangle^S$ corresponds to $f^S_{maj}$ that only considers the "defined" grades in the computation of the median. Indeed, consider an incomplete profile $\overline{P^S}$ and denote by $P'_i=g(\overline{P}^S_i)$, then ${f}_{maj}(P'_i) = \rho(P')_{\floor{\frac{\card{P'}}{2}} + 1}$, $\forall i \in S$.

To summarize, starting with zero knowledge, we ask the voters to judge $k$ candidates. Given the partial information at our disposal, we are able to define the "known" median grade for each alternative. We can now use the \emph{selection function} $F^S_k$, defined in \Cref{sec:complete}, to select the $k$ alternatives with the highest "known" median grades.

Given the set $\tilde{K}=F^S_k(g(\overline{P}^S_i))$ of the best $k$ alternatives, every voter is then asked to judge all the candidates in $\tilde{K}$. 
This process results in a restriction $P^{\tilde{K}}$ of the complete profile $P$. It is important to mention that when we ask the voters to judge an alternative $i\in \tilde{K}$ we assume that they report their preference as they would have stated it when asked about $P$. In other words, $P^{\tilde{K}}_{i} = P_i$ for any $i \in \tilde{K}$.

Please note that $P^{\tilde{K}}$ is a complete matrix of $kn$ grades and that we fall back to the complete profile case, thus, we apply the \emph{majority-grade}, $f^{\tilde{K}}_{maj}$, function to $P^{\tilde{K}}$ to determine the median grades and then the winner function $F^{\tilde{K}}_{maj}$ to select the winner. 
%For simplicity we denote by $\overline{W}_{\overline{P^k}} \subseteq A$ the results of this process.

\begin{remark}
	Because we are interested into investigating \acs{MJ}, we are going to use an alphabet with the same size of the one proposed by \citet{Balinski2011} which is composed of the following adjectives: Excellent, Very good, Good, Average, Mediocre, Inadequate, To be Rejected. For brevity we are gonna rename those adjectives respectively from $\delta_7$, corresponding to Excellent, to$\delta_1$, corresponding to To be Rejected. Therefore, $\triangle=\{\delta_1,\delta_2, \delta_3,\delta_4,\delta_5,\delta_6,\delta_7\}$ 
\end{remark}

Once having defined the notation, the first question that comes to mind is about the risks of incomplete knowledge. Given a profile which we only partially know, would we always select the same set of alternatives as winners in case of complete and incomplete knowledge?
\commentOC{The phrasing of this question could be improved: as stated, the answer is, I suppose, obviously negative. With low knowledge, there is not much hope that the right alternatives would be selected.}

\begin{proposition}
	\label{prop:notsamewinner}
	Given $A$ a set of $m\geq 3$ alternatives and $N$ a set of $n\geq3$ voters, there exist a complete profile $P$ and an incomplete profile $\overline{P} \subset P$, such that $F_{maj}(P) \nsubseteq F^{\tilde{K}}_{maj}(P^{\tilde{K}})$ \textemdash where $\tilde{K}=F^S_k(g(\overline{P}^S_i))$.
	\commentOC{As proof I’d suggest: assume we never ask about some alternative $i$ which is judged excellent by everybody; and all the other ones are judged the worst.}
\end{proposition}
\begin{proof}
	Pick an alternative $i\in A$ and a voter $j \in N$, let us build a profile $P$ in the following way: the voter $j$ judges $\delta_1$ the alternative $i$ and she judges $\delta_2$ all the other alternatives $x \in A \setminus \{i\}$; every other voter $y\in N \setminus \{j\}$ judges $\delta_7$ the alternative $i$ and $\delta_6$ all the other alternatives $x \in A \setminus \{i\}$. The resulting profile $P$ will have the form:
	\begin{center}
		$
		\begin{array}{ccccccc}
			& j_1 & j_2 & \dots & j & \dots & j_n \\
			i_1 &	\delta_6 & \delta_6 & \dots & \delta_2 & \dots & \delta_6 \\
			i_2 &	\delta_6 & \delta_6 & \dots & \delta_2 & \dots & \delta_6 \\
			. &	\delta_6 & \delta_6 & \dots & \delta_2 & \dots & \delta_6 \\
			i &	\delta_7 & \delta_7 & \dots & \delta_1 & \dots & \delta_7 \\
			. &	\delta_6 & \delta_6 & \dots & \delta_2 & \dots & \delta_6 \\
			i_m &	\delta_6 & \delta_6 & \dots & \delta_2 & \dots & \delta_6 \\
		\end{array} \quad.
		$
	\end{center}
	For $n\geq 3$ the median grade $f_{maj}(P_i)=\delta_7$ and $f_{maj}(P_x)=\delta_6, \forall x \in A \setminus \{i\}$. Thus, the winner $F_{maj}(P)=\{i\}$.
	
	Consider now any $k \in \intvl{1,m-1}$, and remove from $P$ as many judgments as necessary such that the following conditions are verified: each column only contains $k$ grades \textemdash meaning that each voter judges $k$ candidates \textemdash and the row corresponding to the alternative $i$ is only formed by two grades, one $\delta_7$ and one $\delta_1$ \textemdash we have only the best and worst opinion on $i$. We fill all the unknown judgments with \textit{Undefined} grades to obtain $\overline{P}$. Note that we only have two judgments for the alternative $i$ so $\overline{f}_{maj}(\overline{P}_i)=\delta_1$ and, since $m\geq3$ and $1 \leq k \leq m-1$, $\overline{f}_{maj}(\overline{P}_x)$ is either $\delta_6$ or $\delta_2$ $\forall x \in A \setminus \{i\}$. Therefore $\overline{f}_{maj}(\overline{P}_x) > \overline{f}_{maj}(\overline{P}_i)$ $\forall x \in A \setminus \{i\}$, this means that $i$ will not be selected for the second turn, when we ask the voters their complete preferences about the $k$ "best" alternatives. Thus $i \notin \tilde{K}$ and, consequently, $F_{maj}(P) \nsubseteq F^{\tilde{K}}_{maj}(P^{\tilde{K}})$.
\end{proof}

\commentBN{Stopped here.}

\section{Reasoning on incompleteness}
Let us denote by $W^*=F_{maj}(P)$ the set of winners associated to the complete profile $P$, and recall that $\overline{W}_{\overline{P^k}}$ indicates the set of winners associated to the incomplete profile $\overline{P^k}$.

As we observed in \Cref{prop:notsamewinner}, given an incomplete profile $\overline{P^k} \subset P$ it is possible to elect a candidate as a winner that would not be selected when considering (one of) its completion $P \in C(\overline{P^k})$. In this section we want to investigate how likely it is to happen.

Assume that the complete profile $P$ exists but it is unknown to us, the knowledge we have is represented by $\overline{P^k}$ and the voters are truthful. 
\commentOC{I would not repeat this, it is clear already, and it is strange to repeat it here and not elsewhere. And, you’d need to define truthful.}
This means that starting from $\overline{P^k}$, by asking each voter to judge every alternative we converge to $P$.

Consider an alternative $w\in A$ and observe that for $w$ to be in $W^*$ we need $f_{maj}(P_w)\geq f_{maj}(P_i), \ \forall i \in A$. 
Consider now the incomplete profile $\overline{P^k}$ and the set of the $k$ alternatives with the highest "known" median grades $\tilde{K}$. 
\commentOC{Do you mean, the set $\tilde{K}$ of the $k$ alternatives…?}
Let $v$ be the $k$-th highest median, i.e. $v=\min_{i\in \tilde{K}} \overline{f}_{maj}(\overline{P_i})$ the lowest grade of an alternative in $\tilde{K}$. If $\overline{f}_{maj}(\overline{P}_w) \geq v$ then $w \in \tilde{K}$. 
\commentOC{A notation for the $x$th median grade or $x$th best alternative would be useful.}\commentBN{ok}

Because the voters express their judgments on all the alternatives in $\tilde{K}$ 
\commentOC{The reason for this claim is unclear to me. Is this a deduction, or an hypothesis?}\commentBN{Need more clarity, this should be clear.}
and $P^{\tilde{K}}_{i} = P_i$ for any $i \in \tilde{K}$, then if $w \in W^*$ and $w \in \tilde{K}$ then $w \in \overline{W}_{\overline{P^k}}$.

In order to compute the probability of a winning alternative $w$ to not be elected in case of incompleteness, we need to focus on the probability for $w$ not to reach the second round: probability of $w \notin \tilde{K}$.
\commentBN{If I'm the winner here are three cases I'm not elected in incompleteness:
\begin{itemize}
	\item I'm demoted $\Rightarrow \overline{h}<h$
	\begin{itemize}
		\item others are promoted $\Rightarrow v>\overline{h}$
		\item or not $\Rightarrow v>\overline{h}$
	\end{itemize}
	\item $\overline{h}=h$
	\begin{itemize}
		\item others are promoted $\Rightarrow v>\overline{h}$
	\end{itemize}
	\item I'm promoted $\Rightarrow \overline{h}>h$
	\begin{itemize}
		\item others are promoted $\Rightarrow v>\overline{h}$
	\end{itemize}
\end{itemize}}

Consider the complete profile $P$ and the highest median grade $h=\max_{i\in A}\rho(P_i)_{\floor{\frac{n}{2}} + 1}$. Since $w \in W^*$ then $f_{maj}(P_{w})\geq h$ and the vector of grades $P_w$ must be composed of at least $\floor{\frac{n}{2}}+1$ grades $\alpha \geq h$ and at most $\lceil \frac{n}{2}\rceil-1$ of grades $\beta < h$. Let us evaluate the worst case scenario.
Without loss of generality consider $n$ an odd value, then the vector $P_{w}$ in the worst case contains exactly $\frac{n+1}{2}$ grades greater than or equal to $h$ and $\frac{n-1}{2}$ grades lower than $h$.
\[P_w : \qquad [ \alpha_1, \dots , \alpha_{\frac{n+1}{2}}, \beta_1, \dots , \beta_{\frac{n-1}{2}} ] \]
Consider now the incomplete profile $\overline{P^k} \subset P$ and the highest "known" median $v$ described above. For $w \notin \tilde{K}$ then $\overline{f}_{maj}(\overline{P^k}_w) < v$ and the partial vector $\overline{P^k}_w$ of $x\in \intvl{1,n-1}$ defined grades must be composed of at least $\frac{x+1}{2}$ grades $\beta'<v$ and at most $\frac{x-1}{2}$ grades $\alpha' \geq v$. 

If $h=v$, we have $\binom{(n+1)/2}{(x-1)/2}$ ways of picking $\frac{x-1}{2}$ grades greater than or equal to $v$ out of the $\frac{n+1}{2}$ of the original vector; and $\binom{(n-1)/2}{(x+1)/2}$ ways of picking $\frac{x+1}{2}$ grades lower than $v$.
\begin{align}
	P_{w}: \qquad [ \underbrace{\alpha_1, \dots , \alpha_{\frac{n+1}{2}}}_{\begin{pmatrix}\frac{n+1}{2} \\ \frac{x-1}{2}\end{pmatrix}}, \underbrace{\beta_1, \dots , \beta_{\frac{n-1}{2}}}_{\begin{pmatrix}\frac{n-1}{2} \\ {\frac{x+1}{2}}\end{pmatrix}} ] \\
	\overline{P^k}_w:\qquad [ \overbrace{\alpha'_1, \dots , \alpha'_{\frac{x-1}{2}}}, \overbrace{\beta'_1, \dots , \beta'_{\frac{x+1}{2}}}]
\end{align} 
\newcommand{\largemath}[1]{{\mathlarger{\mathlarger{\mathlarger{\mathlarger{\mathlarger#1}}}}}}
%note to myself: find a better way please

We define the probability of $w \notin \tilde{K}$ as the number of incomplete vectors $\overline{P^k}_w$ for which $\overline{f}_{maj}(\overline{P^k}_w) < v$, over the total number of possible incomplete vectors:
\commentBN{prob of w to be demoted in worst case}
\[ \largemath{\sum}_{x=1}^{n-1}{ \frac{ \largemath{\sum}_{i=0}^{x/2}{ \begin{pmatrix}\frac{n+1}{2} \\ {\frac{x-1}{2}-i}\end{pmatrix} \cdot \begin{pmatrix}\frac{n-1}{2} \\ {\frac{x+1}{2}+i}\end{pmatrix} }}{\begin{pmatrix}n \\ x\end{pmatrix}} } \]

\commentBN{prob of w to be demoted for $\beta$ going from $j=0$ to $(n-1)/2$}

\[ P(\overline{h}<h)= \largemath{\sum}_{j=0}^{(n-1)/2}{ \ \largemath{\sum}_{x=1}^{n-1}{ \frac{ \largemath{\sum}_{i=0}^{x/2}{ \begin{pmatrix}\frac{n+1}{2}+j \\ {\frac{x-1}{2}-i}\end{pmatrix} \cdot \begin{pmatrix}\frac{n-1}{2}-j \\ {\frac{x+1}{2}+i}\end{pmatrix} }}{\begin{pmatrix}n \\ x\end{pmatrix}} }} \]

\commentBN{prob of w to be promoted for $\alpha$ going from $j=0$ to $(n-1)/2$ where: the vector of grades $P_w$ must be composed of at least $\floor{\frac{n}{2}}+1$ grades $\alpha > h$ and at most $\lceil \frac{n}{2}\rceil-1$ of grades $\beta \leq h$.}

\[ P(\overline{h}>h)= \largemath{\sum}_{j=0}^{(n-1)/2}{ \ \largemath{\sum}_{x=1}^{n-1}{ \frac{ \largemath{\sum}_{i=0}^{x/2}{ \begin{pmatrix}\frac{n-1}{2}-j \\ {\frac{x+1}{2}+i}\end{pmatrix} \cdot \begin{pmatrix}\frac{n+1}{2}+j \\ {\frac{x-1}{2}-i}\end{pmatrix} }}{\begin{pmatrix}n \\ x\end{pmatrix}} }} \]

\commentBN{$P(\overline{h}=h)= 1-P(\overline{h}>h)\cdot P(\overline{h}<h)$ : probability of being properly labeled}

The size $x$ of an incomplete vector $\overline{P_i}$ for $i \in A$, depends on the number of questions $k$ asked to the voters, in fact, if we ask $n$ voters to judge $k$ random alternatives, ideally, each alternative $i$ will be judged $\frac{k\cdot n}{m}$ times. Consider the value $k$ as a function of the number of alternatives: $k=c \cdot m$ where $c \in \R$. If $m=10$ and $k=5$ then $k=1/2 m$, i.e. we ask the voters to judge half of the candidates. Thus, the value $x$ depends only on the number of voters $n$ $x=\frac{k\cdot n}{m}= c \cdot n$, $c\in \R$. Without loss of generality we consider both $x$ and $n$ odd values.

Consider the worst case scenario: the real vector $P_{w^*}$ has a proportion of $51\%-49\%$ of $\alpha-\beta$ grades. \Cref{fig:differentX51-49} shows the probability of electing a non-real winner in this scenario for different size $x$ of the incomplete vector $\overline{P_{w^*}}$. \Cref{tab:differentX51-49} shows in details those values. Note that when $x=1001$ the probability is about $25\%$, but we should keep in mind that $x= c \cdot n$, thus a vector of size $1000$ means that the alternative $w^*$ was judged by only $1/10$ of the voters. With this in mind, we see that with $x=\frac{n}{2}$ we obtain a very low probability of only $2.12\%$, but we need $4/5$ of the voters to judge each alternative to get zero probability of "miss-qualification".

The situation change drastically for different proportions of $\alpha-\beta$ grades as \Cref{fig:differentX} shows. In particular, we only need about $200$ judgments (thus $1/50$n) to reach an almost zero probability of electing a non real winner when the real vector $P_{w^*}$ has a proportion of $60\% \alpha -40\% \beta$ grades. Recalling the formula:
\begin{align}
	x&=\frac{k \cdot n}{m} \\
	200&=\frac{k}{m}\cdot 10000 \\
	\frac{1}{50}&=\frac{k}{m} \\
	k&=\frac{m}{50}
\end{align}
we note that asking one question per voter is more than enough to avoid the election of a non-real winner.

In the 2016 elections organised by LaPrimaire.org $n=10675$ voters participated in the first round, and each of them judged $k=5$ random alternatives out of the $m=12$ total ones. Each alternative received an average of $4449$ judgments. Using this data, we simulated th probability of electing a non-real winner for different proportions of $\alpha-\beta$ grades. \Cref{fig:original} and \Cref{tab:original} show the results.

By crossing these results we note that we could have asked the voters far less than $5$ questions, reducing the communication and the cognitive cost of the elicitation process.

%xticklabel style = {font=\footnotesize},
%x label style={at={(axis description cs:0.5,-0.05)},anchor=north}
\begin{figure}
	\centering
	\begin{tikzpicture}
		\begin{axis}[
			ylabel=Prob. \%,
			xlabel= x,
			ymin=0,
			ymax=50,
			xmin=1,
			xmax=10001,
			xtick={1,1001,2001,3001,4001,5001,6001,7001,8001,9001,10001},
			xticklabels={$10^{-3}$,1,2,3,4,5,6,7,8,9,10},
			xticklabel style = {yshift=-0.5ex},
			scaled x ticks= real:1000,
			x label style={at={(axis description cs:0.5,-0.03)},anchor=north}
			]
			\addplot[thick, blue] table [x=x, y=ProbOfMiss, col sep=comma]{data/51-49-100.csv};			
		\end{axis}
	\end{tikzpicture}
	\caption{Probability of electing a non-real winner, for different values of $x$, with $n=10000$, and $51\%-49\%$ proportion of $\alpha - \beta$ grades.}
	\label{fig:differentX51-49}
\end{figure}

\sisetup{table-number-alignment = center, table-figures-integer=2, table-figures-decimal=1, table-auto-round}
\begin{table}
	\centering
	\begin{tabular}{S[table-figures-integer=5, table-figures-decimal=0]S[table-figures-integer=2, table-figures-decimal=2]}
			\toprule
			{x} & {Prob. of Miss} \\
			\midrule
			1	&	48.9853044087	\\
			1001	&	24.9190117413	\\
			2001	&	15.5009678852	\\
			3001	&	9.1920240364	\\
			4001	&	4.8710050444	\\
			5001	&	2.1180123415	\\
			6001	&	0.645530701	\\
			7001	&	0.096388706	\\
			8001	&	0.0024354987	\\
			9001	&	0.000000051	\\
			10001	&	0.00	\\
			\bottomrule
		\end{tabular}
	\caption{Detailed numbers of \Cref{fig:differentX51-49}.}
	\label{tab:differentX51-49}
\end{table}

\begin{figure}
	\centering
	\begin{tikzpicture}
		\begin{axis}[
			ylabel=Prob. \%,
			xlabel= x,
			ymin=0,
			ymax=50,
			xmin=1,
			xmax=201,
			enlarge x limits=-1, %hack to plot on the full x-axis scale
			width=13cm, %set bigger width
			height=6cm,
			legend style={font=\scriptsize}
			]
			\addlegendimage{mark=*,teal,mark size=1.5}
			\addlegendimage{mark=triangle*,orange,mark size=1.5}
			\addlegendimage{mark=square*,blue,mark size=1.5}
			\addlegendimage{mark=diamond*,red,mark size=1.5}
			
			\addplot[thick, mark=*, mark size = {2}, mark indices = {15}, teal] table [x=x, y=ProbOfMiss, col sep=comma]{data/60-40-2.csv};
			\addlegendentry{$60\%-40\%$}
			\addplot[thick, mark=triangle*, mark size = {2}, mark indices = {6}, orange] table [x=x, y=ProbOfMiss, col sep=comma]{data/70-30-2.csv};
			\addlegendentry{$70\%-30\%$}	
			\addplot[thick, mark=square*, mark size = {2}, mark indices = {4}, blue] table [x=x, y=ProbOfMiss, col sep=comma]{data/80-20-2.csv};	
			\addlegendentry{$80\%-20\%$}
			\addplot[thick, mark=diamond*, mark size = {2}, mark indices = {2}, red] table [x=x, y=ProbOfMiss, col sep=comma]{data/90-10-2.csv};			
			\addlegendentry{$90\%-10\%$}
		\end{axis}
	\end{tikzpicture}
	\caption{Probability of electing a non-real winner, for different values of $x$ and different proportion of $\alpha - \beta$ grades, with $n=10000$.}
	\label{fig:differentX}
\end{figure}


\begin{figure}
	\centering
	\begin{tikzpicture}
		\begin{axis}[
			ylabel=Prob. of Miss \%,
			xlabel=Percentage of $\alpha$ Grades \%,
			ymin=0,
			ymax=5,
			xmin=5445,
			xmax=9608,
			scaled ticks = false,
			xtick={5445,6405,7473,8540,9608},
			xticklabels={51,60,70,80,90}
			]
			\addplot[thick, red] table [x=BetterThanMed, y=ProbOfMiss, col sep=comma]{data/original.csv};			
		\end{axis}
	\end{tikzpicture}
	\caption{Probability of electing a non-real winner, for $n=10675$, $x=4449$ and different proportion of $\alpha - \beta$ grades.}
	\label{fig:original}
\end{figure}

\begin{table}
	\centering
	\begin{tabular}{cc}
		\toprule
		{$\alpha-\beta$} & {Prob. of Miss} \\
		\midrule
		$51\%-49\%$	&	3.92	\\
		$60\%-40\%$	&	2.79$10^{-69}$	\\
		$70\%-30\%$	&	5.80$10^{-318}$	\\
		$80\%-20\%$	&	0.00	\\
		$90\%-10\%$	&	0.00	\\
		\bottomrule
	\end{tabular}
	\caption{Detailed numbers of \Cref{fig:original}.}
	\label{tab:original}
\end{table}

\subsection{Unfixed k}

A natural question that comes to mind when considering the process of asking the voters to judge random alternatives is: how feasible is it? Especially when applying it to political elections, it is safe to say that voters have strong opinions. There are always some candidates that we would never want to see in office, while we would really like to support our favorite candidate. By applying the random selection of questions there is a chance we do not get to express our opinions on those particular candidates. In the worst case, we may be asked to judge only candidates of whom we do not have a strong opinion, or worse, that we do not even know. Is our judgment relevant in this case? How willing are we to take the risk to go and vote without the certainty of being able to express the judgments we consider important?

Because of all these reasons, we may want to consider the possibility for the voters to choose the candidates to judge. One extreme situation that may occur is that each voter judges only its best and worst choice. 

\begin{proposition}
	Given two integers $k=5$ and $s=5$, $m$ alternatives and $n$ voters who only judge their best and worst candidates, there exist a complete profile $P'$ and an incomplete profile $\overline{P'}$ such that $P'\in C(\overline{P'})$ and $F(P')\neq F(\overline{P'})$.
\end{proposition}	

\begin{proof} Consider the following complete profile $P$:

	\begin{center}
		\begin{tabular}{cccccccc}
			& j$_1$ & j$_2$ & j$_3$ \\
			a	&	Average	&	Average	&	Excellent	\\
			b	&	To be rejected	&	Good	&	Good	\\
			c	&	Mediocre	&	Excellent	&	Mediocre	\\
			d	&	Average	&	Average	&	To be rejected	\\
			e	&	Mediocre	&	To be rejected	&	Mediocre	\\
			f	&	Excellent	&	Inadequate	&	Inadequate \\
		\end{tabular}
	\end{center}
	
	
	for the sake of the example the rows are not ordered vectors because the identity of the voters is considered.
	
	The vector of medians $f_{maj}(P)$ is:
	\begin{center}
		$
		\begin{array}{cc}
			a &	Average \\
			b &	Good \\
			c &	Mediocre \\
			d &	Average	\\
			e &	Mediocre \\
			f & Inadequate \\
		\end{array} \quad.
		$
	\end{center}
	The real winner is $F^P=b$. 
	
	Assume that each voter only express its best and worst judgments and construct the complete profile $P'$ from $P$ in the following way: for each alternative $i$ that is not the real winner $w^*$ add as many voters as needed such that its known median grade ($f_{maj}(\overline{P'_i})$) is better than the known median grade of the real winner ($f_{maj}(\overline{P'_{w^*}})$); then add an additional alternative that is rejected by all these new voters. Since the voters only express the best and the worst grades, we are not interested in how they judge the rest of the alternatives, to construct a complete profile we can assume that they judge them according to the current known median. The resulting complete profile $P'$ is the following, but our information $\overline{P'}$ is only restricted to the green values: 

		\scalebox{0.75}{
			\begin{tabular}{cccccccc}
				& j$_1$ & j$_2$ & j$_3$ & j$_4$ & j$_5$ & j$_6$ & j$_7$ \\
				a	&	Average	&	Average	&	{\color{teal}Excellent}	&	Average	&	Average	&	Average	&	Average	\\
				b	&	{\color{teal}To be rejected}	&	Good	&	Good	&	Good	&	Good	&	Good	&	Good	\\
				c	&	Mediocre	&	{\color{teal}Excellent}	&	Mediocre	&	Mediocre	&	Mediocre	&	Mediocre	&	Mediocre	\\
				d	&	Average	&	Average	&	{\color{teal}To be rejected}	&	{\color{teal}Excellent}	&	{\color{teal}Excellent}	&	Average	&	Average	\\
				e	&	Mediocre	&	{\color{teal}To be rejected}	&	Mediocre	&	Mediocre	&	Mediocre	&	{\color{teal}Excellent}	&	{\color{teal}Excellent}	\\
				f	&	{\color{teal}Excellent}	&	Inadequate	&	Inadequate	&	Inadequate	&	Inadequate	&	Inadequate	&	Inadequate	\\
				g	&	{\color{teal}To be rejected}	&	{\color{teal}To be rejected} & {\color{teal}To be rejected}& {\color{teal}To be rejected} & {\color{teal}To be rejected} & {\color{teal}To be rejected} & {\color{teal}To be rejected}	\\
			\end{tabular}
		}
		
	The vector of medians $f_{maj}(\overline{P'})$ is:
	\begin{center}
		$
		\begin{array}{cc}
			a &	\text{Excellent} \\
			b &	\text{To be rejected} \\
			c &	\text{Excellent} \\
			d &	\text{Excellent}	\\
			e &	\text{Excellent} \\
			f & \text{Excellent} \\
			g & \text{To be rejected} \\
		\end{array} \quad.
		$
	\end{center}
	The real winner $b$ does not appear in the set of candidates for the second round $S=\{a,c,d,e,f\}$, so it will not be elected from the incomplete profile $\overline{P'}$.
\end{proof}
	



\newpage
\paragraph{My draft - do not read}
\begin{itemize}
	\item Does expressing judgment on randomly selected candidates influence the result? (If we change the questions does the result change?)
	\item Does the number of questions influence the result? (If we change the number of questions does the result change?)
	\item If yes, do these effects are mitigated by a second round?
	\item Which is the right number of questions? (Best trade-off between communication cost and optimal result.)
	\item Can we select the next question with minimax regret instead of randomly selecting a candidate?
	\item Can we say anything about the "fairness" of proposing the candidates to judge? Suppose I have strong opinions about only two candidates: one I extremely like and one I extremely dislike. There is a chance I will not be asked about those two candidates, in this case I cannot say much about the other candidates and I am also frustrated because I did not get to express my opinions.
	\item Consider $n$ voters and $m$ candidates and assume that a voter $i \in N$ judges only a fraction of the $m$ candidates. What is the resulting voting rule? What are its properties? Can a voter manipulate the result by judging only some candidates? 
\end{itemize}

\newpage
\bibliography{biblio}
\newpage
\appendix
\section{Old material that can be transformed into examples}
\begin{proof} Consider $n=3, m=6, k=5$ and the following complete profile $P$:
	\begin{center}
		$
		\begin{array}{cccc}
			& j_1 & j_2 & j_3 \\
			a &	Excellent	& Excellent & Inadequate\\
			b &	Mediocre	& Mediocre	& Mediocre\\
			c &	Mediocre	& Mediocre & Inadequate\\
			d &	Average	& Average	& Average\\
			e &	Average	& Mediocre	& Inadequate \\
			f &	Average	& Mediocre & Mediocre	  \\
		\end{array} \quad.
		$
	\end{center}
	The vector of medians $f_{maj}(P)$ is:
	\begin{center}
		$
		\begin{array}{cc}
			a &	Excellent \\
			b &	Mediocre \\
			c &	Mediocre \\
			d &	Average	\\
			e &	Mediocre \\
			f & Mediocre \\
		\end{array} \quad.
		$
	\end{center}
	The real winner is $F^P=a$. 
	
	Consider now the following incomplete profiles $\overline{P}$ and $\overline{P}'$ obtained after having asked each voter to judge $k=5$ random chosen alternatives: \commentOC{This mixes again the process and the maths. The proposition does not talk about randomness and it is confusing to refer to this here. The proposition holds whatever the way $P$ bar is chosen (including, deterministically).}\commentBN{I'm not sure I got this. With randomness I mean the one in the definition of $\overline{P}$.}
	\begin{center}
		$\overline{P}: \qquad
		\begin{array}{cccc}
			& j_1 & j_2 & j_3 \\
			a &	Excellent	& {\color{red}Undefined} & Inadequate\\
			b &	Mediocre	& Mediocre	& Mediocre\\
			c &	Mediocre	& Mediocre & Inadequate\\
			d &	Average	& Average	& {\color{red}Undefined} \\
			e &	Average	& Mediocre	& Inadequate \\
			f &	{\color{red}Undefined}	& Mediocre & Mediocre	  \\
		\end{array} \quad,
		$
	\end{center}
	\begin{center}
		$\overline{P}': \qquad
		\begin{array}{cccc}
			& j_1 & j_2 & j_3 \\
			a &	Excellent	& Excellent & Inadequate\\
			b &	Mediocre	& {\color{red}Undefined}	& Mediocre\\
			c &	Mediocre	& Mediocre & {\color{red}Undefined}\\
			d &	Average	& Average	& Average \\
			e &	{\color{red}Undefined}	& Mediocre	& Inadequate \\
			f &	Average	& Mediocre & Mediocre	  \\
		\end{array} \quad.
		$
	\end{center}
	The vector of medians are:
	\begin{center}
		$f_{maj}(\overline{P})= \quad
		\begin{array}{cc}
			a &	Inadequate \\
			b &	Mediocre \\
			c &	Mediocre \\
			d &	Average	\\
			e &	Mediocre \\
			f & Mediocre \\
		\end{array} \quad,\quad%
		f_{maj}(\overline{P}')= \quad
		\begin{array}{cc}
			a &	Excellent \\
			b &	Mediocre \\
			c &	Mediocre \\
			d &	Average	\\
			e &	Inadequate \\
			f & Mediocre \\
		\end{array} \quad.
		$
	\end{center}
	
	Consider the sets of $s=5$ alternatives with the highest median grades for the two profiles, $S'=\{b,c,d,e,f\}$ for $\overline{P}$, and $S^{\prime\prime}=\{a,b,c,d,f\}$ for $\overline{P}'$, and the two restrictions $P_{S'}$ and $P_{S^{\prime\prime}}$. In particular, $P_{S'}$ corresponds to the complete profile when eliminating the alternative $a$, and $P_{S^{\prime\prime}}$ to the complete profile without the alternative $e$.
	The vector of medians are:
	\begin{center}
		$f_{maj}(P_S')= \quad
		\begin{array}{cc}
			b &	Mediocre \\
			c &	Mediocre \\
			d &	Average	\\
			e &	Mediocre \\
			f & Mediocre \\
		\end{array} \quad,\quad%
		f_{maj}(P_S^{\prime\prime})= \quad
		\begin{array}{cc}
			a & Excellent \\
			b &	Mediocre \\
			c &	Mediocre \\
			d &	Average	\\
			f & Mediocre \\
		\end{array} \quad.
		$
	\end{center}
	The winner associated to the incomplete profile $\overline{P}$ is then $F^{P_{S'}} = d$ and the one associated to $\overline{P}'$ is $F^{P_{S^{\prime\prime}}} = a$, thus $F^{\overline{P}} \neq F^{\overline{P}'}$.
\end{proof}
\commentOC{Perhaps some part of this could be transformed to an example.}

\begin{corollary}
	Given $m$ alternatives, $n$ voters and an integer $k \in \intvl{1,m}$, there exist a profile $P$ and an incomplete profile of $P$, $\overline{P}$, such that $F^{\overline{P}}$ is not the real winner.
\end{corollary}
\commentOC{“real winner” is inappropriate here. Is there an unreal winner? I realize that you mean “winner considering the complete profile” VS “winner considering some part of it”, but I don’t think that the term “real” is appropriate. I’d simply say “winners” for the winners of the complete election, and perhaps “approximate winners” for the winners given a partial profile, or something similar.}

\end{document}

