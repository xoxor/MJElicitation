\RequirePackage[l2tabu, orthodox]{nag}
\documentclass[version=3.21, pagesize, twoside=off, bibliography=totoc, DIV=calc, fontsize=12pt, a4paper]{scrartcl}
\input{preamble/packages}
\input{preamble/redac}
\input{preamble/math_basics}
%Decision Theory (MCDA and SC)
\NewDocumentCommand{\allalts}{}{\mathscr{A}}
\NewDocumentCommand{\allcrits}{}{\mathscr{C}}
\NewDocumentCommand{\alts}{}{A}
\NewDocumentCommand{\dm}{}{i}
\NewDocumentCommand{\allF}{}{\mathscr{F}}
\NewDocumentCommand{\allvoters}{}{\mathscr{N}}
\NewDocumentCommand{\voters}{}{N}
\NewDocumentCommand{\prof}{}{\boldsymbol{P}}
\NewDocumentCommand{\linors}{}{\mathscr{L}(\allalts)}
%Thanks to https://tex.stackexchange.com/q/154549
	%\makeatletter
	%\def\@myRgood@#1#2{\mathrel{R^X_{#2}}}
	%\def\myRgood{\@ifnextchar_{\@myRgood@}{\mathrel{R^X}}}
	%\makeatother
\NewDocumentCommand{\pref}{}{\succ}
\NewDocumentCommand{\prefi}{O{i}}{\succ_{#1}}

%Deliberated Judgment
\NewDocumentCommand{\allargs}{}{S^*}
\NewDocumentCommand{\args}{}{S}
\NewDocumentCommand{\ar}{}{s}
\NewDocumentCommand{\allprops}{}{T}
\NewDocumentCommand{\prop}{}{t}
\NewDocumentCommand{\ileadsto}{}{⇝}
\NewDocumentCommand{\ibeatse}{}{⊳_\exists}
\NewDocumentCommand{\nibeatse}{}{⋫_\exists}
\NewDocumentCommand{\ibeatsst}{}{⊳_\forall}
\NewDocumentCommand{\nibeatsst}{}{⋫_\forall}
\NewDocumentCommand{\mleadsto}{O{\eta}}{⇝_{#1}}
\NewDocumentCommand{\mbeats}{O{\eta}}{⊳_{#1}}
\NewDocumentCommand{\ibeatseinv}{}{⊳_\exists^{-1}}

%Logic
\NewDocumentCommand{\ltru}{}{\texttt{T}}
\NewDocumentCommand{\lfal}{}{\texttt{F}}

\NewDocumentCommand{\dbar}{}{\overline{\delta}}
\NewDocumentCommand{\reddbar}{}{{\color{red}\overline{\delta}}}
\NewDocumentCommand{\second}{}{\prime\prime}
\NewDocumentCommand{\Pbar}{}{\overline{P}}
\NewDocumentCommand{\fmaj}{o}{
	\IfNoValueTF {#1}{f_\mathit{maj}}{f^{#1}_\mathit{maj}}
}
\NewDocumentCommand{\Fmaj}{o}{
	\IfNoValueTF {#1}{F_\mathit{maj}}{F^{#1}_\mathit{maj}}
}
\NewDocumentCommand{\fmajbar}{o}{
	\IfNoValueTF {#1}{\overline{f}_\mathit{maj}}{\overline{f}^{#1}_\mathit{maj}}
}
\NewDocumentCommand{\Fmajbar}{o}{
	\IfNoValueTF {#1}{\overline{F}_\mathit{maj}}{\overline{F}^{#1}_\mathit{maj}}
}
\NewDocumentCommand{\med}{}{\text{med}}

%\input{preamble/draw}
%\input{preamble/jdoc}
\usepackage{stmaryrd}
\usepackage{xcolor}
\usepackage{siunitx}
\usepackage{relsize}
%I find these settings useful in draft mode. Should be removed for final versions.
	%Which line breaks are chosen: accept worse lines, therefore reducing risk of overfull lines. Default = 200.
		\tolerance=2000
	%Accept overfull hbox up to...
		\hfuzz=2cm
	%Reduces verbosity about the bad line breaks.
		\hbadness 5000
	%Reduces verbosity about the underful vboxes.
		\vbadness=1300

\title{Preference Elicitation under Majority Judgment}
\author{}
%\title{The title \thanks{Thanks.}}
%\author{Olivier Cailloux}
%\author{Name2}
%\affil{Université Paris-Dauphine, Université PSL, CNRS, LAMSADE, 75016 PARIS, FRANCE\\
%	\href{mailto:olivier.cailloux@dauphine.fr}{olivier.cailloux@dauphine.fr}
%}
%\author{Name3}
%\affil{Affil2}
%\hypersetup{
%	pdfsubject={},
%	pdfkeywords={},
%}

\begin{document}
\maketitle

\begin{abstract}
\acl{MJ} (\acs{MJ}) is a voting system where voters assign grades to candidates using an ordinal scale. The winner is the candidate with the highest majority-grade \textemdash which is the median of the grades received. This method has attracted increasing attention of french associations and political parties which have started to use \acs{MJ} for internal decisions or local elections. In particular LaPrimaire.org is a french association that uses \acs{MJ} to choose its candidate for the french presidential election. The vote is conducted in two rounds: in the first one the voters judge five candidates randomly picked; the five candidates with the highest medians pass at the second round as finalists and the voters are asked to judge them. Is the random selection of candidates a good elicitation technique? In this paper we explore the consequences of profile incompleteness and we question the elicitation of voters preferences.
\end{abstract}

\section{Introduction}
\label{sec:intro}
\acl{MJ} (\acs{MJ}) is a voting method proposed by \citet{Balinski2007,Balinski2011} to elect one out of $m$ candidates based on the judgments of $n$ voters. The latter express their preferences by assigning to each candidate one of the following adjectives: Excellent, Very good, Good, Average, Mediocre, Inadequate, To be rejected. Those adjectives represent a common language whose semantic is assumed to be a shared knowledge among the voters carrying thus an absolute meaning. For each candidate the median of the grades she received is computed, this is called \textit{majority-grade}. The candidate with the highest majority-grade is elected. Ties are broken by considering the majority-grade of first order: one vote associated with the majority-grade of each tied candidates is removed and their medians are recomputed. The candidate with the highest new median is elected. If there is still a tie the process is repeated until a unique winner is found. The authors describe an additional tie breaking procedure that uses the \textit{majority-gauge}, but \citet{Felsenthal2008} show that it does not always yield the same result as the iterative mechanism.

\subsection{Related work}
The idea of using the median in voting is not new, the first use can be traced back to Galton's 'middlemost' \citep{Galton1907a,Galton1907b}. More recently \citet{Bassett1999} proposed the median as a substitute for Borda's mean, advocating for its statistical robustness \textemdash which measures the sensitivity to departures from the hypothesized model.

Numerous observers described the median grade as the highest level at which a candidate obtains the support of the majority of the voters. In other words, starting for the highest grade $h$ we check if the majority of the voters assigned at least $h$ to some candidate $c$. If this is not the case, we descend in the grading scale until such level $\hat{h}$ is found where a candidate $\hat{c}$ satisfies half population. The grade $\hat{h}$ is then the median of $\hat{c}$ grades, and, since it is the first level we stopped at, it corresponds to the best possible median. This method was proposed by James W. Bucklin in the early twentieth century \citep{Hoag1926} and it was rediscovered several times in literature for example under the names of \textit{Majoritarian Compromise} \citep{Sertel1986,Sertel1999} and \textit{Fallback Bargaining} \citep{Brams2001}. Note, also, that when the number of grades is equal to two (approve, disapprove) then \acs{MJ} is reduced to Approval Voting. Several authors studied the strengths and the weakness of this method \citep{Felsenthal2008,Laslier2018} and Balinski replied to most of the critics in an article written in french published on the Revue économique \citep{Balinski2019}.


\subsection{Where is it used?}
\acs{MJ} has being adopted by a progressively larger number of french political parties including: Le Parti Pirate, Génération(s), LaPrimaire.org, France Insoumise and La République en Marche.
%https://www.lopinion.fr/edition/politique/en-marche-teste-elections-jugement-majoritaire-mode-scrutin-tres-201884
"Mieux Voter" \citep{MV} is a french association that promotes the use of \acs{MJ} as voting method whenever a collective choice has to be selected: public administration, associations, companies. On their website it is possible to find all the citizens lists \textendash party lists that are not affiliated to any national political party \textemdash that used \acs{MJ} to rank their candidates during the local elections of 2020. In two cases, Bordeaux et Annecy, the candidate selected using \acs{MJ} was then elected as a mayor. 


\subsection{Case LaPrimaire.org}
\label{sec:primaire}
LaPrimaire.org \citep{LaPrimaire} is a french political initiative whose goal is to select an independent candidate for the french presidential election using \acs{MJ} as voting rule. All french citizens over 18 with rights to vote can participate as candidates or voters. The association Democratech implemented the platform for the first time in 2016 in view of the 2017 presidential elections. The number of voters who participated in the election was $10676$ during the first round (with $53383$ votes) and $32685$ during the second round (with $163425$ votes). Between May and October 2021 the process will be repeated to select the candidate who will run for the 2022 presidential elections \citep{LaPrimaire2022}.

The procedure consists of several steps whose duration is defined by a calendar. In the first phase, any eligible candidate can submit her nomination to the platform and the voters can support one or multiple nominations. The candidacies that receive at least 500 supports pass to the next phase and represent the candidates for the first round of the election. In the first round each voter is asked to express her judgment, using \acs{MJ}, on five random candidates. At the end of this phase the five candidates with the highest medians are considered the finalists who qualify for the second round. In the second round each voter is asked to express her judgment, using \acs{MJ}, on all the five finalists. The candidate with the best median at the end of this phase is selected as representative for the presidential election.

It is important to mention that the participation of this candidate to the actual election is not granted. In fact, by the french law a candidate must collect at least 500 signatures of elected officials in order to participate to the presidential election. The candidate selected by the voters of LaPrimaire.org in 2016 collected only 135 signatures and did not participate in the 2017 presidential elections.  

\section{Notation}
Let us formally define the problem. Consider a finite set $N$ of voters (or judges) with $\#N=n\geq 2$ and a finite set $A$ of alternatives (or competitors) with $\#A=m\geq 3$. 
A \textit{common language} $\Delta = \{ \alpha, \beta, \dots \}$ is a set of strictly ordered grades. It may, or may not, be finite and the notation $\alpha \geq \beta$ indicates that $\alpha$ is a better or equivalent grade than $\beta$. A profile $P = P(A,N) = \Delta^{m \times n}$ is a $m$ by $n$ matrix of grades. A row $P_i$, $i\in \intvl{1,m}$, represents the ordered vector of grades $(r_1 , \dots, r_n )$ associated to the alternative $i$. The double brackets denote an interval in the integers. 

A grading function $f: P \rightarrow \Delta^m$ is a function that assigns to a profile $P$ a vector of final grades, one for each alternative. In \citet{Balinski2007} definition, in order to be a social grading function, $f$ must be neutral, anonymous, unanimous, monotonic, independent of irrelevant alternatives and continuous. 
% the \emph{middlemost} aggregation function $f$, for each vector of grades $r_i= (r_1 , \dots, r_n )$ associated to the alternative $i \in \intvl{1,m}$, returns: 
%\begin{align}
%	f(r_i) &= r_{(n+1)/2} \text{ when n is odd,} \\
%	r_{n/2} \geq f(r_i) &\geq r_{(n+2)/2} \text{ otherwise.}
%\end{align}
The \emph{majority-grade}, $f_{maj}$, is a social grading function that associates to each row $i$ of the profile $P$ its median grade value. With a small abuse of notation we can define the value $f_{maj}(P_i)$, i.e. the median grade of the alternative $i$, as:
\begin{align}
	f_{maj}(P_i) = \begin{cases}
		r_{(n+1)/2} \text{ when n is odd,} \\
		r_{(n+2)/2} \text{ otherwise.}
	\end{cases}
\end{align}
%i.e. it corresponds to the lower middlemost.
% It could also be written as $f_{maj}(P_i) = \lfloor \frac{n}{2} \rfloor + 1$.
The winner function $F:\Delta^{m} \rightarrow i\in A$ is a function that selects the alternative with the highest median grade as a winner: $w^* = \argmax_{i\in A}f_{maj}(P_i)$. In case two or more alternatives have the same highest median grade $\alpha$, the tie is broken by removing one $\alpha$ grade from each of the tied alternatives, recomputing the new median grade and repeating the process until one unique winner is found. We will refer with $F^P$ the result of the function $F(f_{maj}(P))$.

\subsection{Incomplete Profile}
In order to analyse the study case described in \Cref{sec:primaire} we need to adapt the mechanism to incomplete profiles. Let $\bar{P} \subseteq P$ be our knowledge about the profile $P$. %The voters have full knowledge of their own judgments that we need to elicit starting from zero knowledge.
The starting knowledge is represented by a matrix $m\times n$ of \textit{Undefined} grades. This is an additional grade that will not count in the computation of the median grade. We define with $K_j \in A$ a set of $k$ random alternatives that we ask the voter $j\in N$ to judge. The resulting incomplete profile obtained from repeating this process for every $j \in N$, is a matrix $k \times n$ of grades, $\Delta^{k\times n}$, and its said to be $k$-sized. Note that when $k=n$ we obtain the complete profile.

The \emph{majority-grade} for incomplete profile $f_{\bar{maj}}: \bar{P} \rightarrow \Delta^m$ is a function that assigns to an incomplete profile $\bar{P}$ a vector of final grades, one for each alternative. The \emph{first-step selection function} $F':\Delta^{m} \rightarrow S \subseteq A$ is a function that selects the $s \in \intvl{1,m}$ alternatives with the highest median grades as candidates for the second phase. All the voters are then asked to judge the alternatives in $S$ and we denote this restriction of the complete profile as $P_{S}$. Note that we fall back to the complete profile case, thus, we apply the \emph{majority-grade}, $f_{maj}$, function to $P_{S}$ to determine the median grades and then the winner function $F$ to select the winner.

\begin{definition}
	We define the real winner $w^*$ of a profile $P$ as the result of $F^P$ when $P$ is complete, i.e. $w^* = \argmax_{i\in A}f_{maj}(P_i)$.
\end{definition}

\begin{proposition}
	\label{prop:notsamewinner}
	Given $m$ alternatives, $n$ voters and an integer $k \in \intvl{1,m}$, there exists a profile $P$ and two $k$-sized incomplete profiles of $P$, $\bar{P}$ and $\bar{P}'$, such that $F^{\bar{P}} \neq F^{\bar{P}'}$.
\end{proposition}

\begin{proof} Consider $n=3, m=6, k=5$ and the following complete profile $P$:
	\begin{center}
		$
		\begin{array}{cccc}
			  & j_1 & j_2 & j_3 \\
			a &	Excellent	& Excellent & Inadequate\\
			b &	Mediocre	& Mediocre	& Mediocre\\
			c &	Mediocre	& Mediocre & Inadequate\\
			d &	Average	& Average	& Average\\
			e &	Average	& Mediocre	& Inadequate \\
			f &	Average	& Mediocre & Mediocre	  \\
		\end{array} \quad.
		$
	\end{center}
	The vector of medians $f_{maj}(P)$ is:
	\begin{center}
		$
		\begin{array}{cc}
			a &	Excellent \\
			b &	Mediocre \\
			c &	Mediocre \\
			d &	Average	\\
			e &	Mediocre \\
			f & Mediocre \\
		\end{array} \quad.
		$
	\end{center}
	The real winner is $F^P=a$. 
	
	Consider now the following incomplete profiles $\bar{P}$ and $\bar{P}'$ obtained after having asked each voter to judge $k=5$ random chosen alternatives:
	\begin{center}
		$\bar{P}: \qquad
		\begin{array}{cccc}
			& j_1 & j_2 & j_3 \\
			a &	Excellent	& {\color{red}Undefined} & Inadequate\\
			b &	Mediocre	& Mediocre	& Mediocre\\
			c &	Mediocre	& Mediocre & Inadequate\\
			d &	Average	& Average	& {\color{red}Undefined} \\
			e &	Average	& Mediocre	& Inadequate \\
			f &	{\color{red}Undefined}	& Mediocre & Mediocre	  \\
		\end{array} \quad,
		$
	\end{center}
	\begin{center}
		$\bar{P}': \qquad
		\begin{array}{cccc}
			& j_1 & j_2 & j_3 \\
			a &	Excellent	& Excellent & Inadequate\\
			b &	Mediocre	& {\color{red}Undefined}	& Mediocre\\
			c &	Mediocre	& Mediocre & {\color{red}Undefined}\\
			d &	Average	& Average	& Average \\
			e &	{\color{red}Undefined}	& Mediocre	& Inadequate \\
			f &	Average	& Mediocre & Mediocre	  \\
		\end{array} \quad.
		$
	\end{center}
	The vector of medians are:
	\begin{center}
		$f_{maj}(\bar{P})= \quad
		\begin{array}{cc}
			a &	Inadequate \\
			b &	Mediocre \\
			c &	Mediocre \\
			d &	Average	\\
			e &	Mediocre \\
			f & Mediocre \\
		\end{array} \quad,\quad%
		f_{maj}(\bar{P}')= \quad
		\begin{array}{cc}
			a &	Excellent \\
			b &	Mediocre \\
			c &	Mediocre \\
			d &	Average	\\
			e &	Inadequate \\
			f & Mediocre \\
		\end{array} \quad.
		$
	\end{center}

	Consider the sets of $s=5$ alternatives with the highest median grades for the two profiles, $S'=\{b,c,d,e,f\}$ for $\bar{P}$, and $S^{\prime\prime}=\{a,b,c,d,f\}$ for $\bar{P}'$, and the two restrictions $P_{S'}$ and $P_{S^{\prime\prime}}$. In particular, $P_{S'}$ corresponds to the complete profile when eliminating the alternative $a$, and $P_{S^{\prime\prime}}$ to the complete profile without the alternative $e$.
	The vector of medians are:
	\begin{center}
		$f_{maj}(P_S')= \quad
		\begin{array}{cc}
			b &	Mediocre \\
			c &	Mediocre \\
			d &	Average	\\
			e &	Mediocre \\
			f & Mediocre \\
		\end{array} \quad,\quad%
		f_{maj}(P_S^{\prime\prime})= \quad
		\begin{array}{cc}
			a & Excellent \\
			b &	Mediocre \\
			c &	Mediocre \\
			d &	Average	\\
			f & Mediocre \\
		\end{array} \quad.
		$
	\end{center}
	The winner associated to the incomplete profile $\bar{P}$ is then $F^{P_{S'}} = d$ and the one associated to $\bar{P}'$ is $F^{P_{S^{\prime\prime}}} = a$, thus $F^{\bar{P}} \neq F^{\bar{P}'}$.
\end{proof}

\begin{corollary}
	Given $m$ alternatives, $n$ voters and an integer $k \in \intvl{1,m}$, there exist a profile $P$ and an incomplete profile of $P$, $\bar{P}$, such that $F^{\bar{P}}$ is not the real winner.
\end{corollary}

\subsection{Electing a non-real winner}

In this section we want to investigate the probability of electing an alternative different from the real winner when considering an incomplete profile.

Consider $\bar{P_i}$ the partial vector of ordered grades associated to the alternative $i\in A$, let $C(\bar{P_i}) = \{P_i \in \Delta^n | \bar{P_i} \subseteq P_i\}$ denote the set of possible completions of $\bar{P_i}$. The set $C(\bar{P}) = \prod_{i \in A} C(\bar{P_i})$ represents the set of complete profiles extending $\bar{P}$; note that $P \in C(\bar{P})$. We define the best completion of a partial vector $\bar{P_i}$ the completion $\hat{P_i}\in C(\bar{P_i})$ such that $f_{maj}(\hat{P_i})\geq f_{maj}(P_i'), \ \forall P_i' \in C(\bar{P_i})$.

The only way for a real winner $w^*$ not to get elected is to not be part of the set of candidates for the second phase $S$. In fact, suppose $w^*\in S$ and its partial median $f_{maj}(\bar{P_{w^*}})$ is lower than the one of another candidate $i\in A\setminus\{w^*\}$; in the second phase the complete profile restricted to the alternatives in $S$ is built and because $w^*$ is a real winner, then $f_{maj}(P_{S, w^*})\geq f_{maj}(P_{S, i}), \ \forall i \in A$. Thus, a real winner will always win in the second phase. 

Consider the $s$ alternatives with the highest median grades in $\bar{P}$, in particular, let $v$ be the $s$-th highest median, i.e. $v=\max_{i\in S}f_{maj}(\bar{P_i})$ . For $w^*$ not to be in $S$ it must be $f_{maj}(\bar{P_{w^*}})<v$. Since $w^*$ is a real winner then $f_{maj}(P_{w^*})>v$, and the grades vector of $w^*$ must be composed of at least $\lfloor \frac{n}{2}\rfloor+1$ grades $\alpha>v$ and at most $\lceil \frac{n}{2}\rceil-1$ of grades $\beta<v$. Without loss of generality consider $n$ an odd value, then the vector $P_{w^*}$ in the worst case contains exactly $\frac{n+1}{2}$ grades greater than $v$ and $\frac{n-1}{2}$ grades lower than $v$. Thus, for $f_{maj}(\bar{P_{w^*}})$ to be smaller than $v$, the partial vector $\bar{P_{w^*}}$ of size $x$ must be composed of at least $\frac{x+1}{2}$ grades $\beta<v$ and at most $\frac{x-1}{2}$ grades $\alpha>v$. 
%Try to fix the bynomial parenthesis 
\begin{align}
	P_{w^*}: \qquad [ \underbrace{\alpha_1, \dots , \alpha_{\frac{n+1}{2}}}_{\begin{pmatrix}\frac{n+1}{2} \\ \frac{x-1}{2}\end{pmatrix}}, \underbrace{\beta_1, \dots , \beta_{\frac{n-1}{2}}}_{\begin{pmatrix}\frac{n-1}{2} \\ {\frac{x+1}{2}}\end{pmatrix}} ] \\
	\bar{P_{w^*}}:\qquad [ \overbrace{\alpha_1, \dots , \alpha_{\frac{x-1}{2}}}, \overbrace{\beta_1, \dots , \beta_{\frac{x+1}{2}}}]
\end{align} 

\newcommand{\largemath}[1]{{\mathlarger{\mathlarger{\mathlarger{\mathlarger{\mathlarger#1}}}}}}
%note to myself: find a better way please
We define the probability of electing a non-real winner as the probability of $w^*$ not being in $S$. The latter can be defined as the number of incomplete vectors $\bar{P_{w^*}}$ of size $x$ for which $f_{maj}(\bar{P_{w^*}})<v$, over the total number of possible incomplete vectors of $x$ elements:
\[ \frac{ \largemath{\sum}_{i=0}^{x/2}{ \begin{pmatrix}\frac{n+1}{2} \\ {\frac{x-1}{2}-i}\end{pmatrix} \cdot \begin{pmatrix}\frac{n-1}{2} \\ {\frac{x+1}{2}+i}\end{pmatrix} }}{\begin{pmatrix}n \\ x\end{pmatrix}} \]

The size $x$ of an incomplete vector $\bar{P_i}$ for $i \in A$, depends on the number of questions $k$ asked to the voters, in fact, if we ask $n$ voters to judge $k$ random alternatives, ideally, each alternative $i$ will be judged $\frac{k\cdot n}{m}$ times. Consider the value $k$ as a function of the number of alternatives: $k=c \cdot m$ where $c \in \R$. If $m=10$ and $k=5$ then $k=1/2 m$, i.e. we ask the voters to judge half of the candidates. Thus, the value $x$ depends only on the number of voters $n$ $x=\frac{k\cdot n}{m}= c \cdot n$, $c\in \R$. Without loss of generality we consider both $x$ and $n$ odd values.

Consider the worst case scenario: the real vector $P_{w^*}$ has a proportion of $51\%-49\%$ of $\alpha-\beta$ grades. \Cref{fig:differentX51-49} shows the probability of electing a non-real winner in this scenario for different size $x$ of the incomplete vector $\bar{P_{w^*}}$. \Cref{tab:differentX51-49} shows in details those values. Note that when $x=1001$ the probability is about $25\%$, but we should keep in mind that $x= c \cdot n$, thus a vector of size $1000$ means that the alternative $w^*$ was judged by only $1/10$ of the voters. With this in mind, we see that with $x=\frac{n}{2}$ we obtain a very low probability of only $2.12\%$, but we need $4/5$ of the voters to judge each alternative to get zero probability of "miss-qualification".

The situation change drastically for different proportions of $\alpha-\beta$ grades as \Cref{fig:differentX} shows. In particular, we only need about $200$ judgments (thus $1/50$n) to reach an almost zero probability of electing a non real winner when the real vector $P_{w^*}$ has a proportion of $60\% \alpha -40\% \beta$ grades. Recalling the formula:
\begin{align}
	x&=\frac{k \cdot n}{m} \\
	200&=\frac{k}{m}\cdot 10000 \\
	\frac{1}{50}&=\frac{k}{m} \\
	k&=\frac{m}{50}
\end{align}
we note that asking one question per voter is more than enough to avoid the election of a non-real winner.

In the 2016 elections organised by LaPrimaire.org $n=10675$ voters participated in the first round, and each of them judged $k=5$ random alternatives out of the $m=12$ total ones. Each alternative received an average of $4449$ judgments. Using this data, we simulated th probability of electing a non-real winner for different proportions of $\alpha-\beta$ grades. \Cref{fig:original} and \Cref{tab:original} show the results.

By crossing these results we note that we could have asked the voters far less than $5$ questions, reducing the communication and the cognitive cost of the elicitation process.

%xticklabel style = {font=\footnotesize},
%x label style={at={(axis description cs:0.5,-0.05)},anchor=north}
\begin{figure}
	\centering
	\begin{tikzpicture}
		\begin{axis}[
			ylabel=Prob. \%,
			xlabel= x,
			ymin=0,
			ymax=50,
			xmin=1,
			xmax=10001,
			xtick={1,1001,2001,3001,4001,5001,6001,7001,8001,9001,10001},
			xticklabels={$10^{-3}$,1,2,3,4,5,6,7,8,9,10},
			xticklabel style = {yshift=-0.5ex},
			scaled x ticks= real:1000,
			x label style={at={(axis description cs:0.5,-0.03)},anchor=north}
			]
			\addplot[thick, blue] table [x=x, y=ProbOfMiss, col sep=comma]{data/51-49-100.csv};			
		\end{axis}
	\end{tikzpicture}
	\caption{Probability of electing a non-real winner, for different values of $x$, with $n=10000$, and $51\%-49\%$ proportion of $\alpha - \beta$ grades.}
	\label{fig:differentX51-49}
\end{figure}

\sisetup{table-number-alignment = center, table-figures-integer=2, table-figures-decimal=1, table-auto-round}
\begin{table}
	\centering
	\begin{tabular}{S[table-figures-integer=5, table-figures-decimal=0]S[table-figures-integer=2, table-figures-decimal=2]}
			\toprule
			{x} & {Prob. of Miss} \\
			\midrule
			1	&	48.9853044087	\\
			1001	&	24.9190117413	\\
			2001	&	15.5009678852	\\
			3001	&	9.1920240364	\\
			4001	&	4.8710050444	\\
			5001	&	2.1180123415	\\
			6001	&	0.645530701	\\
			7001	&	0.096388706	\\
			8001	&	0.0024354987	\\
			9001	&	0.000000051	\\
			10001	&	0.00	\\
			\bottomrule
		\end{tabular}
	\caption{Detailed numbers of \Cref{fig:differentX51-49}.}
	\label{tab:differentX51-49}
\end{table}

\begin{figure}
	\centering
	\begin{tikzpicture}
		\begin{axis}[
			ylabel=Prob. \%,
			xlabel= x,
			ymin=0,
			ymax=50,
			xmin=1,
			xmax=201,
			enlarge x limits=-1, %hack to plot on the full x-axis scale
			width=13cm, %set bigger width
			height=6cm,
			legend style={font=\scriptsize}
			]
			\addlegendimage{mark=*,teal,mark size=1.5}
			\addlegendimage{mark=triangle*,orange,mark size=1.5}
			\addlegendimage{mark=square*,blue,mark size=1.5}
			\addlegendimage{mark=diamond*,red,mark size=1.5}
			
			\addplot[thick, mark=*, mark size = {2}, mark indices = {15}, teal] table [x=x, y=ProbOfMiss, col sep=comma]{data/60-40-2.csv};
			\addlegendentry{$60\%-40\%$}
			\addplot[thick, mark=triangle*, mark size = {2}, mark indices = {6}, orange] table [x=x, y=ProbOfMiss, col sep=comma]{data/70-30-2.csv};
			\addlegendentry{$70\%-30\%$}	
			\addplot[thick, mark=square*, mark size = {2}, mark indices = {4}, blue] table [x=x, y=ProbOfMiss, col sep=comma]{data/80-20-2.csv};	
			\addlegendentry{$80\%-20\%$}
			\addplot[thick, mark=diamond*, mark size = {2}, mark indices = {2}, red] table [x=x, y=ProbOfMiss, col sep=comma]{data/90-10-2.csv};			
			\addlegendentry{$90\%-10\%$}
		\end{axis}
	\end{tikzpicture}
	\caption{Probability of electing a non-real winner, for different values of $x$ and different proportion of $\alpha - \beta$ grades, with $n=10000$.}
	\label{fig:differentX}
\end{figure}


\begin{figure}
	\centering
	\begin{tikzpicture}
		\begin{axis}[
			ylabel=Prob. of Miss \%,
			xlabel=Percentage of $\alpha$ Grades \%,
			ymin=0,
			ymax=5,
			xmin=5445,
			xmax=9608,
			scaled ticks = false,
			xtick={5445,6405,7473,8540,9608},
			xticklabels={51,60,70,80,90}
			]
			\addplot[thick, red] table [x=BetterThanMed, y=ProbOfMiss, col sep=comma]{data/original.csv};			
		\end{axis}
	\end{tikzpicture}
	\caption{Probability of electing a non-real winner, for $n=10675$, $x=4449$ and different proportion of $\alpha - \beta$ grades.}
	\label{fig:original}
\end{figure}

\begin{table}
	\centering
	\begin{tabular}{cc}
		\toprule
		{$\alpha-\beta$} & {Prob. of Miss} \\
		\midrule
		$51\%-49\%$	&	3.92	\\
		$60\%-40\%$	&	2.79$10^{-69}$	\\
		$70\%-30\%$	&	5.80$10^{-318}$	\\
		$80\%-20\%$	&	0.00	\\
		$90\%-10\%$	&	0.00	\\
		\bottomrule
	\end{tabular}
	\caption{Detailed numbers of \Cref{fig:original}.}
	\label{tab:original}
\end{table}

\subsection{Unfixed k}

A natural question that comes to mind when considering the process of asking the voters to judge random alternatives is: how feasible is it? Especially when applying it to political elections, it is safe to say that voters have strong opinions. There are always some candidates that we would never want to see in office, while we would really like to support our favorite candidate. By applying the random selection of questions there is a chance we do not get to express our opinions on those particular candidates. In the worst case, we may be asked to judge only candidates of whom we do not have a strong opinion, or worse, that we do not even know. Is our judgment relevant in this case? How willing are we to take the risk to go and vote without the certainty of being able to express the judgments we consider important?

Because of all these reasons, we may want to consider the possibility for the voters to choose the candidates to judge. One extreme situation that may occur is that each voter judges only its best and worst choice. 

\begin{proposition}
	Given two integers $k=5$ and $s=5$, $m$ alternatives and $n$ voters who only judge their best and worst candidates, there exist a complete profile $P'$ and an incomplete profile $\bar{P'}$ such that $P'\in C(\bar{P'})$ and $F(P')\neq F(\bar{P'})$.
\end{proposition}	

\begin{proof} Consider the following complete profile $P$:

	\begin{center}
		\begin{tabular}{cccccccc}
			& j$_1$ & j$_2$ & j$_3$ \\
			a	&	Average	&	Average	&	Excellent	\\
			b	&	To be rejected	&	Good	&	Good	\\
			c	&	Mediocre	&	Excellent	&	Mediocre	\\
			d	&	Average	&	Average	&	To be rejected	\\
			e	&	Mediocre	&	To be rejected	&	Mediocre	\\
			f	&	Excellent	&	Inadequate	&	Inadequate \\
		\end{tabular}
	\end{center}
	
	
	for the sake of the example the rows are not ordered vectors because the identity of the voters is considered.
	
	The vector of medians $f_{maj}(P)$ is:
	\begin{center}
		$
		\begin{array}{cc}
			a &	Average \\
			b &	Good \\
			c &	Mediocre \\
			d &	Average	\\
			e &	Mediocre \\
			f & Inadequate \\
		\end{array} \quad.
		$
	\end{center}
	The real winner is $F^P=b$. 
	
	Assume that each voter only express its best and worst judgments and construct the complete profile $P'$ from $P$ in the following way: for each alternative $i$ that is not the real winner $w^*$ add as many voters as needed such that its known median grade ($f_{maj}(\bar{P'_i})$) is better than the known median grade of the real winner ($f_{maj}(\bar{P'_{w^*}})$); then add an additional alternative that is rejected by all these new voters. Since the voters only express the best and the worst grades, we are not interested in how they judge the rest of the alternatives, to construct a complete profile we can assume that they judge them according to the current known median. The resulting complete profile $P'$ is the following, but our information $\bar{P'}$ is only restricted to the green values: 

		\scalebox{0.75}{
			\begin{tabular}{cccccccc}
				& j$_1$ & j$_2$ & j$_3$ & j$_4$ & j$_5$ & j$_6$ & j$_7$ \\
				a	&	Average	&	Average	&	{\color{teal}Excellent}	&	Average	&	Average	&	Average	&	Average	\\
				b	&	{\color{teal}To be rejected}	&	Good	&	Good	&	Good	&	Good	&	Good	&	Good	\\
				c	&	Mediocre	&	{\color{teal}Excellent}	&	Mediocre	&	Mediocre	&	Mediocre	&	Mediocre	&	Mediocre	\\
				d	&	Average	&	Average	&	{\color{teal}To be rejected}	&	{\color{teal}Excellent}	&	{\color{teal}Excellent}	&	Average	&	Average	\\
				e	&	Mediocre	&	{\color{teal}To be rejected}	&	Mediocre	&	Mediocre	&	Mediocre	&	{\color{teal}Excellent}	&	{\color{teal}Excellent}	\\
				f	&	{\color{teal}Excellent}	&	Inadequate	&	Inadequate	&	Inadequate	&	Inadequate	&	Inadequate	&	Inadequate	\\
				g	&	{\color{teal}To be rejected}	&	{\color{teal}To be rejected} & {\color{teal}To be rejected}& {\color{teal}To be rejected} & {\color{teal}To be rejected} & {\color{teal}To be rejected} & {\color{teal}To be rejected}	\\
			\end{tabular}
		}
		
	The vector of medians $f_{maj}(\bar{P'})$ is:
	\begin{center}
		$
		\begin{array}{cc}
			a &	\text{Excellent} \\
			b &	\text{To be rejected} \\
			c &	\text{Excellent} \\
			d &	\text{Excellent}	\\
			e &	\text{Excellent} \\
			f & \text{Excellent} \\
			g & \text{To be rejected} \\
		\end{array} \quad.
		$
	\end{center}
	The real winner $b$ does not appear in the set of candidates for the second round $S=\{a,c,d,e,f\}$, so it will not be elected from the incomplete profile $\bar{P'}$.
\end{proof}
	



\newpage
\paragraph{My draft - do not read}
\begin{itemize}
	\item Does expressing judgment on randomly selected candidates influence the result? (If we change the questions does the result change?)
	\item Does the number of questions influence the result? (If we change the number of questions does the result change?)
	\item If yes, do these effects are mitigated by a second round?
	\item Which is the right number of questions? (Best trade-off between communication cost and optimal result.)
	\item Can we select the next question with minimax regret instead of randomly selecting a candidate?
	\item Can we say anything about the "fairness" of proposing the candidates to judge? Suppose I have strong opinions about only two candidates: one I extremely like and one I extremely dislike. There is a chance I will not be asked about those two candidates, in this case I cannot say much about the other candidates and I am also frustrated because I did not get to express my opinions.
	\item Consider $n$ voters and $m$ candidates and assume that a voter $i \in N$ judges only a fraction of the $m$ candidates. What is the resulting voting rule? What are its properties? Can a voter manipulate the result by judging only some candidates? 
\end{itemize}

\newpage
\bibliography{biblio}

\end{document}

